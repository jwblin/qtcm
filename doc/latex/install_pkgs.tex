% ==========================================================================
% Python packages
%
% By Johnny Lin
% ==========================================================================


% ------ BODY -----
%

The following Python packages are required to be installed on your
system in a directory listed in your \vars{sys.path}:
\begin{itemize}
\item \htmladdnormallinkfoot{Python}%
	{http://www.python.org/}:  The Python programming language
	and interpreter.  Make sure you have a version recent enough
	to be compatible with all the needed Python packages.
\item \htmladdnormallinkfoot{\mods{matplotlib}}%
	{http://matplotlib.sourceforge.net/}:  Scientific plotting
	package, using Matlab-like syntax.  The \mods{basemap} toolkit
	for \mods{matplotlib} must also be installed.
\item \htmladdnormallinkfoot{NumPy}%
	{http://numpy.scipy.org/}:  The standard array package for
	Python.  The module name of NumPy imported in a Python 
	session is \mods{numpy}.
\item \htmladdnormallinkfoot{Scientific Python}%
	{http://dirac.cnrs-orleans.fr/plone/software/scientificpython/}:
	Has netCDF file operators, in addition to other routines
	of use in scientific computing.  The module name of
	Scientific Python imported in a Python session is
	\mods{Scientific}.
\end{itemize}

One other required Python package, \mods{f2py}, is now a part of the
NumPy package, and so installation of NumPy is sufficient to give
you both.

The package \htmladdnormallinkfoot{SciPy}{http://www.scipy.org},
which includes several Python-accessible scientific libraries, also
includes NumPy (and thus \mods{f2py}), so if you install SciPy,
you don't have to install NumPy again.  Note that SciPy is not the
same as Scientific Python; the names are confusing.

A few non-Python packages are also required:
\begin{itemize}
\item \LaTeX: A scientific typesetting program used by the 
	\class{Qtcm} instance method \mods{plotm} to handle 
	exponents and subscripts.  The most common Unix 
	distribution of \LaTeX\ is
	\htmladdnormallinkfoot{teTeX}{http://www.tug.org/teTeX}.

\item netCDF:  This set of libraries enables one to write datasets into
	a platform independent, binary format, with metdata attached.
	The \htmladdnormallinkfoot{netCDF 3.6.2 library}%
        	{http://www.unidata.ucar.edu/software/netcdf/}
	source code can be
\latexhtml{downloaded from UCAR\footnote{http://www.unidata.ucar.edu/downloads/netcdf/netcdf-3\_6\_2/}}%
        {\htmladdnormallink{downloaded from UCAR}{http://www.unidata.ucar.edu/downloads/netcdf/netcdf-3_6_2/}}.
\end{itemize}

For most Unix installations, the easiest way to install all the
above is via a package manager, for instance \mods{apt-get} in
Debian GNU/Linux, \mods{aptitude} or \mods{synaptic} in Ubuntu
GNU/Linux, and \mods{fink} in Mac OS X.  Of course, you can also
download a package's source code and build direct and/or install
using Python's
\htmladdnormallinkfoot{\mods{distutils}}{http://docs.python.org/dist/dist.html}
utilities.




% ===== end of file =====
