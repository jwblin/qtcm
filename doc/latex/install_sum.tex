% ==========================================================================
% Installation Summary
%
% By Johnny Lin
% ==========================================================================


% ------ BODY -----
%

This section provides a summary of the steps needed to install
\mods{qtcm}, and a description of the naming conventions used in
this chapter.  If you have had a decent amount of experience with
Python and installing software on a Unix system, this section will
probably be all you need to read.  The installation steps are:

\begin{enumerate}
\item Install a Fortran compiler (see Section~\ref{sec:fort.compilers}
	for a list of compilers known to work).
	This compiler should be in a directory
	listed in your system path (e.g., \fn{/usr/bin}, etc.).

\item Install all required packages
	(see Section~\ref{sec:py.etc.pkgs} for details):
	Python,
	\mods{matplotlib} (plus the \mods{basemap} toolkit),
	NumPy (which includes \mods{f2py}),
	Scientific Python,
	\LaTeX,
	and
	netCDF.

	Python packages are required to be installed on your
	system in a directory listed in your \vars{sys.path},
	and the other packages/libraries are required to be in 
	standard directories listed in your system path 
	(e.g., \fn{/usr/bin}, \fn{/sw/include}, etc.).

	Make sure the executable for Python can be called at the
	Unix command line by typing both \cmd{python}.
	You might need to define a Unix alias
	that maps \cmd{python2.4} (or whichever version of Python
	you are using) to \cmd{python}.

\item \latexhtml{Download\footnote{http://www.johnny-lin.com/py\_pkgs/qtcm/}}%
        {\htmladdnormallink{Download}{http://www.johnny-lin.com/py_pkgs/qtcm/}}
	the \mods{qtcm} tarball and extract the distribution
	into a temporary directory for building purposes.
	\fn{qtcm-0.1.2}is the name of
	the \mods{qtcm} distribution directory;
	the number following the hyphen is the
	version number of the distribution.  \label{list:download.qtcm.sum}

	In this manual, the path to \fn{qtcm-0.1.2}will
	be called the ``\mods{qtcm} build path'' and be given as
	\fn{/buildpath}.  When you see \fn{/buildpath}, please substitute
	the actual temporary directory you created for building purposes.

\item The \mods{qtcm} distribution directory 
	\fn{qtcm-0.1.2}contains the following 
	principal sub-directories:
	\fn{doc}, \fn{lib}, \fn{src}, \fn{test}.
	Documentation is in \fn{doc},
	all the package modules are in \fn{lib},
	building of extension modules will take place in \fn{src},
	and testing of the package is done in \fn{test}.

\item Compile \mods{qtcm} extension modules in \fn{src}:
	Go to \fn{src}, copy the makefile from
	\fn{src/Makefiles} corresponding to your
	system into \fn{src}, rename to \fn{makefile},
	make changes to the makefile as needed,
	and execute:
	\begin{codeblock}
	\codeblockfont{%
	make clean \\
	make \_qtcm\_full\_365.so \\
	make \_qtcm\_parts\_365.so}
	\end{codeblock}
	If you executed the make commands in \fn{src,},
	the extension modules will be automatically placed in
	\fn{lib} in the \fn{qtcm-0.1.2}directory.
	See Section~\ref{sec:create.so} for details.
	\label{list:compile.so.sum}

\item Copy the entire contents of \fn{lib} in
	\fn{qtcm-0.1.2}(not \fn{lib} itself) 
	to a directory named
	\fn{qtcm} that is on your \mods{sys.path}.  For instance,
	for Mac OS X using Fink,
	many Python packages are located in a directory
	named \fn{/sw/\-lib/\-python2.4/\-site-packages}, or something
	similar, and this directory is on the system \mods{sys.path}.  
	If this is the case for your system, copy the
	contents of \fn{lib} into
	\fn{/sw/lib/\-python2.4/\-site-packages/\-qtcm}.
	(For Unix systems, the equivalent directory is usually
	\fn{/usr/\-local/\-lib/\-python2.4/\-site-packages}.)

\item Test the \mods{qtcm} distribution in \fn{test}:
	This step is optional and can take a while.
	Testing requires you to first generate a suite of benchmarks
	using the pure-Fortran QTCM1 model, then running the tests of
	\mods{qtcm} by typing:
	\begin{codeblock}
	\codeblockfont{%
python test\_all.py}
	\end{codeblock}
	at the Unix command line while in \fn{test}.
	See Section~\ref{sec:test.qtcm} for details.

\end{enumerate}

At some point, I will automate the installation using Python's
\htmladdnormallinkfoot{\mods{distutils}}{http://docs.python.org/dist/dist.html}
utilities.



% ===== end of file =====
