% ==========================================================================
% Using QTCM
%
% By Johnny Lin
% ==========================================================================


% ------ BODY -----
%
%---------------------------------------------------------------------
\section{Introduction}

Now that you've successfully run your first model instances, in
this chapter I provide detailed explanations regarding the features
of \mods{qtcm}.  I present these explanations in a documentary
rather than didactic fashion; my goal is to document how the features
work.  More details are given in the code docstrings.  At the end
of the chapter, in Section~\ref{sec:cookbook}, I provide a few
cookbook ideas/examples of ways to use the model.




%---------------------------------------------------------------------
\section{Model Instances}  \label{sec:model.instances}

An instance of a \class{Qtcm} model is created in \mods{qtcm} the same way
you create an instance of any class.
For instance, to instantiate two \class{Qtcm}
models, \vars{model1} and \vars{model2}, I type the following:

\begin{codeblock}
\codeblockfont{%
from qtcm import Qtcm \\
model1 = Qtcm(compiled\_form\thinspace=\thinspace'full') \\
model2 = Qtcm(compiled\_form\thinspace=\thinspace'parts')}
\end{codeblock}

In the above example, \vars{model1} uses the compiled QTCM1 model
that runs the model (essentially) using the Fortran driver,
while \vars{model2} uses the compiled QTCM1 model where execution
order and content all the way down to the atmospheric timestep level
is controlled by Python run lists.  (Section~\ref{sec:compiledform}
has more details about the difference between compiled forms.)

For each instance of \class{Qtcm}, copies of all needed extension
modules (e.g., \fn{.so} files) are copied to a temporary directory
that is automatically created by Python.  The full path name of
that directory is saved in the instance attribute \vars{sodir}.
These extension modules are then associated with the specific instance 
through private instance attributes,
and thus every instance of \class{Qtcm} has its own separate variable
and name space on both the Fortran and Python sides.\footnote%
	{The private instance attribute is \vars{\_\_qtcm}.
	See Section~\ref{sec:Qtcm.private.attrib} for details about 
	private \class{Qtcm} instance attributes.}
The temporary directory and all of its contents are deleted when the 
model instance is deleted.

On instantiation, \class{Qtcm} instances set all scalar field
variables to their default values as given in the submodule
\mods{defaults} (and listed in Section~\ref{sec:defaults.scalar}),
and assign the fields as instance attributes.  The instance attribute
\vars{init\_with\_instance\_state} is set to True by default, unless
overridden on instantiation.




%---------------------------------------------------------------------
\section{Initializing a Model Run}

In the pure-Fortran QTCM1, there are three broad
classes of initialized variables:
\begin{enumerate}
\item Those that are read-in using a namelist, 
\item Those that the are read-in from a restart file, and
\item Those that are set by assignment in the Fortran code.  
\end{enumerate}
These variables are a combination of scalars and arrays.

For \mods{qtcm}, interfaces were built so that all classes of
initialized variables that could be user-controlled are accessible
and changeable at the Python-level.  For \mods{qtcm},
the set of variables that could be changed is also expanded, to
include not just the first and second classes of pure-Fortran
QTCM1 initialized variables.  This was done to make \mods{qtcm}
more flexible.  All variables that can be passed between the
compiled QTCM1 model and Python model levels are called
field variables, and are described in detail in
Section~\ref{sec:field.variables}.

As it happens, all the namelist-set variables are scalars.  In the
pure-Fortran QTCM1, those variables are given default values prior
to reading in of the namelist.  To duplicate this functionality,
on model instantiation, all scalar fields are set to their default
values as given in the submodule \mods{defaults} and listed in
Section~\ref{sec:defaults.scalar}.  Most of the default values in
\mods{defaults} are the same as in the pure-Fortran QTCM1, but
there are a few differences.\footnote%
	{One difference being \vars{mrestart}, which in \vars{qtcm} 
	will have the value of 0 in contrast to the pure-Fortran 
	QTCM1 where the default is the 1.}
This setting of scalar defaults is the same for both
\vars{compiled\_form\thinspace=\thinspace'full'} and
\vars{compiled\_form\thinspace=\thinspace'parts'} instances.
Of course, all
\mods{qtcm} fields are user-controllable, both via keyword input
parameters at model instantiation as well as through direct
manipulation of the instance attribute that stores the field variable.

The pure-Fortran QTCM1 initialized prognostic variables and
right-hand sides are set in the Fortran subroutine \mods{varinit}.
Their they are read-in from a restart file or, as default,
set by assignment.
In \mods{qtcm}, the same variables are initialized by a \class{Qtcm}
instance method of the same name, \mods{varinit}, for the case when
\vars{compiled\_form\thinspace=\thinspace'parts'}.  For the case
of \vars{compiled\_form\thinspace=\thinspace'full'}, the compiled
QTCM1 subroutine that is the same as in the pure-Fortran case is
used, and that routine is inaccessible at the Python level.
See Section~\ref{sec:snapshots}'s listing of snapshot variables,
which also includes the prognostic variables and right-hand sides that
are set in \mods{varinit} (both Fortran and Python).




%---------------------------------------------------------------------
\section{The \vars{compiled\_form} Keyword}  \label{sec:compiledform}

The \mods{qtcm} package is a Python wrap of the Fortran routines
that make up QTCM1.  The wrapping layer adds flexibility and
functionality, but at the cost of speed.  Thus, I created two
types of extension modules from the Fortran QTCM1 code, one
which permits very little control over the compiled Fortran
\emph{routines} at the Python level, and one that allows the Python-level
to control model execution in the compiled QTCM1 model
all the way down to the atmospheric timestep level.\footnote%
	{That control is via run lists, which are described in
	Section~\ref{sec:runlists}.}
The former extension module corresponds to 
\vars{compiled\_form\thinspace=\thinspace'full'} and
the latter extension module to
\vars{compiled\_form\thinspace=\thinspace'parts'}.

For \vars{compiled\_form\thinspace=\thinspace'full'},
the compiled portion of the model encompasses (nearly) the
entire QTCM1 model as a whole.  Thus, the only compiled QTCM1 model
modules or subroutines that Python should interact with is
the \mods{driver} routine (which executes the entire model) and
the \mods{setbypy} module (which enables communication between the
compiled model and the Python-level of model fields.\footnote%
	{The \mods{setbypy} Python module is the wrap of the
	Fortran QTCM1 \mods{SetByPy} module.}

For \vars{compiled\_form\thinspace=\thinspace'parts'}, the compiled
portion of the model does not encompasses the model as a whole, but
rather is broken up into separate units (as appropriate) all the
way down to an atmosphere timestep.  Thus, compiled QTCM1 model
modules/subroutines that are accessible at the Python-level include
those that are executed within an atmosphere timestep on up.

Because the difference in compiled forms fundamentally affects how
the \class{Qtcm} instance facilitates Python-Fortran communication,
this attribute must be set on instantiation via a keyword input
parameter.

In the rest of this section, to avoid being verbose, when I
write \vars{'full'}, I mean the situation where
\vars{compiled\_form\thinspace=\thinspace'full'}.
Likewise, when I
write \vars{'parts'}, I mean the situation where
\vars{compiled\_form\thinspace=\thinspace'parts'}.


	\subsection{Initialization for 
			\vars{compiled\_form\thinspace=\thinspace'full'}}
				\label{sec:init.compiledform.full}

For a model run of this case, the \class{Qtcm} instance will
initialize the model using the Fortran \mods{varinit} subroutine
in the compiled QTCM1 model.  This subroutine does the following:

\begin{itemize}
\item If \vars{mrestart\thinspace=\thinspace1}, 
	the restart file is used to initialize all prognostic
	variables.  In terms of start date, the following rules are
	used:
	\begin{enumerate}
	\item Variable \vars{dateofmodel} is read from the restart file.
	\item If \vars{day0}, \vars{month0}, and \vars{year0}
		are negative, or otherwise
		invalid (e.g., \vars{month0} greater than 12), the invalid
		value is replaced with the
		day, month, and/or year of the day \emph{after} 
		that given by \vars{dateofmodel}.
		If the value of \vars{day0}, \vars{month0}, or \vars{year0}
		is not invalid in this sense, it is not replaced.
	\end{enumerate}
	Thus, if the restart file gives 
	\vars{dateofmodel} equal to 101102
	(year 10, month 11, day 2), and 
	\vars{day0\thinspace=\thinspace-1}, 
	\vars{month0\thinspace=\thinspace-1}, 
	\vars{year0\thinspace=\thinspace-1},
	and 
	\vars{mrestart\thinspace=\thinspace1}, 
	the model will start running from year 10, month 11, day 3.
	If \vars{dateofmodel} equals to 101102, and 
	\vars{day0\thinspace=\thinspace-1}, 
	\vars{month0\thinspace=\thinspace3}, 
	\vars{year0\thinspace=\thinspace-1},
	the model will start running from year 10, month 3, day 3.

\item If \vars{mrestart\thinspace=\thinspace0}, 
	all prognostic variables and right-hand sides are set to an
	initial value (which for most of those variables is zero).
	In terms of start date, \vars{day0} is set to 1 (and thus 
	the value of \vars{day0} previously input is ignored), and
	both \vars{month0} and \vars{year0}
	are set to 1 
	if their previously input values are invalid (where
	invalid means less than
	1, or, for \vars{month0}, greater than 12).
	Otherwise, \vars{month0} and \vars{year0} are left unchanged.
	Variable \vars{dateofmodel} has the value it had when the variable
	was declared (which is determined by the compiler and usually
	is zero; \vars{dateofmodel} will not be properly set until
	subroutine \mods{TimeManager} is called.

	Thus, if 
	\vars{day0\thinspace=\thinspace-1},
	\vars{month0\thinspace=\thinspace-1}, 
	\vars{year0\thinspace=\thinspace-1} is input into the model
	(say from a namelist) and 
	\vars{mrestart\thinspace=\thinspace0},
	the model will start running from year 1, month 1, day 1,
	and \vars{dateofmodel} at the exit of subroutine 
	\mods{varinit} will equal its compiler-set default.
	If 
	\vars{day0\thinspace=\thinspace14}, 
	\vars{month0\thinspace=\thinspace3}, 
	\vars{year0\thinspace=\thinspace11}, and 
	\vars{mrestart\thinspace=\thinspace0} on input into the
	model,
	the model will start running from year 11, month 3, day 1,
	and \vars{dateofmodel} at the exit of subroutine 
	\mods{varinit} will equal its compiler-set default.

	Note that \vars{dateofmodel}
	can thus be inconsistent with 
	\vars{month0} and \vars{year0} at the
	exit of subroutine \mods{varinit}.
\end{itemize}

This behavior with respect to initializing
the start date is different than in QTCM1 versions 1.0 and 2.1.
Please see the source code from those earlier QTCM1 versions for
details.




	\subsection{Initialization for 
			\vars{compiled\_form\thinspace=\thinspace'parts'}}
				\label{sec:init.compiledform.parts}

For \vars{'parts'} model, the methodology of how initialized
prognostic variables, right-hand sides, and start date related
variables are set is controlled by the \class{Qtcm} instance
attribute/flag \vars{init\_with\_instance\_state}.  The initialization
is (mostly) executed in the Python \vars{varinit} method in the
following way:

\begin{itemize}
\item If \vars{init\_with\_instance\_state} is False:
The method as described for
initialization for the 
\vars{'full'} case is generally
followed, with the exception that dateofmodel is set
to match \vars{day0}, \vars{month0}, \vars{year0}, prior to exit of 
\mods{varinit}.

\item If \vars{init\_with\_instance\_state} is True:
the model object will initialize the model based on the current
state of the model instance.  This enables you to set a model run
session's initial conditions based upon the state of the prognostic
variables and parameters stored at the Python level, which is
accessible at runtime.
\end{itemize}


Since the \vars{init\_with\_instance\_state\thinspace=\thinspace{False}}
case is mainly described by the initialization method for the
\vars{'full'} case, I refer the
reader to Section~\ref{sec:init.compiledform.full}.
For the case of \vars{init\_with\_instance\_state} is True, however,
the task is more complicated.  Specifically, for that case,
initialization includes the following:

\begin{enumerate}
\item If not currently defined,
	variable \vars{dateofmodel} is set to a default value of 0,
	which is specified in the module defaults.

\item The \vars{mrestart} flag is ignored for variable initialization.

\item All prognostic variables and right-hand sides
        are set to an
        initial value (which for most of those variables is zero),
	unless the variable is defined at the Python level, in which
	case the inital value is set to the Python level defined value.

\item If \vars{dateofmodel} is greater than 0, 
	\vars{day0}, \vars{month0}, and \vars{year0} are overwritten
        with values derived from \vars{dateofmodel} 
	in order to set the run to start
	the day \emph{after} \vars{dateofmodel}.

\item If \vars{dateofmodel} is less than or equal to 0, \vars{day0},
	\vars{month0}, and \vars{year0} are set to their respective
	instance attribute values, if valid.  For invalid instance
	attribute values, the invalid \vars{day0}, \vars{month0},
	and/or \vars{year0} is set to 1.

\item Variable \vars{dateofmodel} is recalculated
	and overwritten to match 
	\vars{day0}, \vars{month0}, \vars{year0}, prior to exit of 
	\mods{varinit}.
\end{enumerate}

As a result, for \vars{init\_with\_instance\_state} is True, the
way you indicate to the model that a run session is a brand-new run
is by setting, before the \mods{run\_session} method call,
\vars{dateofmodel} to a value less than or equal to 0, and \vars{day0},
\vars{model0}, and \vars{year0} to the day you want the model to
begin the run session.  To indicate to the model you wish to continue
a run, set \vars{dateofmodel} to the day \emph{before} you want the
model to start running from.

Examples:
\begin{itemize}
\item If \vars{day0\thinspace=\thinspace-1}, 
	\vars{month0\thinspace=\thinspace-1}, 
	\vars{year0\thinspace=\thinspace-1}, and
	\vars{dateofmodel\thinspace=\thinspace0} is input into 
	the model the model will start running from year 1, month 1, day 1,
	and 
	variable \vars{dateofmodel} at the exit of 
	subroutine \mods{varinit}
	will equal 10101.

\item If \vars{day0\thinspace=\thinspace14},
	\vars{month0\thinspace=\thinspace3}, 
	\vars{year0\thinspace=\thinspace11},
	and \vars{dateofmodel\thinspace=\thinspace0} is input into the
	model, the model will start running from year 11, month 3, day 14,
	and 
	variable \vars{dateofmodel} at the exit of 
	subroutine \mods{varinit} will equal
	110314.

\item If \vars{day0\thinspace=\thinspace14},
	\vars{month0\thinspace=\thinspace3}, 
	\vars{year0\thinspace=\thinspace11},
	and \vars{dateofmodel\thinspace=\thinspace341023} is input into the
	model, the model will start running from year 34, month 10, day 24,
	and at the exit of subroutine 
	\mods{varinit}, \vars{dateofmodel} will equal
	341024, with \vars{day0\thinspace=\thinspace24},
	\vars{month0\thinspace=\thinspace10}, and
	\vars{year0\thinspace=\thinspace34}.
\end{itemize}


	\subsection{Communication Between Python and Fortran-Levels}
				\label{sec:comm.py.fort.compiledform}

After initialization, the second major difference between a
\vars{'full'} and \vars{'parts'} model is how and when communication
between the Python and Fortran levels can occur.  For the \vars{'full'}
case, except for the passing in and out of variables before and after
a run session, all variable passing and subroutine calling happens in
the compiled QTCM1 model, with no control at the Python level.
For the \vars{'parts'} case, variables can be passed between the
Python and Fortran-levels at all levels down to the atmospheric
timestep, and many Fortran QTCM1 subroutines can be called from the
Python-level.  


		\subsubsection{Passing Variables}

For all \vars{compiled\_form} cases, variables are passed back and
forth between the Python \class{Qtcm} instance level and the
compiled QTCM1 model Fortran-level using the \class{Qtcm}
instance methods \mods{get\_qtcm1\_item} and \mods{set\_qtcm1\_item}:\footnote%
	{All Fortran routines used to pass variables back and forth are
	defined in the \mods{setbypy} module of the \fn{.so} extension
	module stored in the \class{Qtcm} instance variable \vars{\_\_qtcm}.
	All Fortran wrappers that enable Python to call compiled QTCM1 model
	subroutines are defined in the \mods{wrapcall} module stored in
	the \class{Qtcm} instance variable \vars{\_\_qtcm}.
	These modules are described in detail in 
	Sections~\ref{sec:setbypy} and~\ref{sec:wrapcall}, respectively.}

\begin{itemize}
\item \mods{get\_qtcm1\_item}(\dumarg{key}):
	Returns the value of the field variable given by the string
	\dumarg{key}.  If the compiled QTCM1 model variable given by
	\dumarg{key} is unreadable, the
        custom exception 
	\vars{FieldNotReadableFromCompiledModel} is thrown.
	The value returned is a copy of the value on the Fortran
	side, not a reference to the variable in memory.

\item \mods{set\_qtcm1\_item}:
	Sets the value of a field variable
	in the compiled QTCM1 model \emph{and at the Python-level,}
	automatically overriding any previous value at both levels.
	Thus, calling this method will change/create the \class{Qtcm}
	instance attribute corresponding to the field variable.
        When the compiled QTCM1 model variable is set, a copy of the
        Python value is passed to the Fortran model; the
	variable is \emph{not passed by reference.}
	This value comes from the \mods{set\_qtcm1\_item} calling
	parameter list, \emph{not} from the \class{Qtcm}
        instance attribute corresponding to the field variable.
\end{itemize}

The \mods{set\_qtcm1\_item} method has two calling forms, one with
one argument and the other with two arguments:
\begin{itemize}
\item One argument:  The method is called
	as \mods{set\_qtcm1\_item}(\dumarg{arg}), where \dumarg{arg} 
	is either a string giving the name of the field variable or 
	a \class{Field} instance.

\item Two arguments:  The method is called as
	\mods{set\_qtcm1\_item}(\dumarg{key}, \dumarg{value}), where
	\dumarg{key} is the string giving the name of the field variable
	and \dumarg{value} is the value to set the model field variable to
	(note \dumarg{value} can be a \class{Field} instance).
\end{itemize}
In either calling form, if no value given, the default value as defined
in module \mods{defaults} is used.

Some compiled QTCM1 model variables are not in a state where they
can be set.  An example is a compiled QTCM1 model pointer variable,
prior to the pointer being associated with a target (an attempt
to set would yield a bus error).  In such cases, the
\mods{set\_qtcm1\_item} method will throw a
\vars{FieldNotReadableFromCompiledModel} exception, nothing will
be set in the compiled QTCM1 model, and the Python counterpart
field variable (if it previously existed) would be left unchanged.\footnote%
	{We handle this situation in this way to enable the
	\class{Qtcm} instance to store variables
	even if the compiled model is not yet ready to accept them.}

Examples, typed in at a Python prompt, and
assuming that \vars{model} is a \class{Qtcm} instance:
\begin{itemize}
\item \cmd{dtvalue\thinspace=\thinspace{model.get\_qtcm1\_item('dt')}}:
	Retrieves the value of field variable \vars{dt} (timestep)
	from the compiled QTCM1 Fortran model and sets it to the
	Python variable \vars{dtvalue}.

\item \cmd{model.set\_qtcm1\_item('dt')}:
	Sets the value of field variable \vars{dt}
	in the compiled QTCM1 Fortran model to the default
	value (as given in \mods{defaults}),
	and sets the value of Python attribute \vars{model.dt} also to 
	that default value.  
	Remember that \vars{model.dt} is a \class{Field}
	instance.

\item \cmd{model.set\_qtcm1\_item('dt', 2000.)}:
	Sets the value of field variable \vars{dt}
	in the compiled QTCM1 Fortran model to 2000 (as a real),
	and sets the value of Python attribute \vars{model.dt} also to 2000.
\end{itemize}


		\subsubsection{Calling Compiled QTCM1 Model Subroutines}

All compiled QTCM1 model subroutines that can be called
(except \mods{driver} and \mods{varptrinit}) are in the
\mods{setbypy} or \mods{wrapcall} modules
of the \class{Qtcm} instance private attribute \vars{\_\_qtcm}.
(On \class{Qtcm} instance instantiation, \vars{\_\_qtcm} is set
to the \fn{.so} extension module that is the compiled QTCM1 Fortran model.)
Thus, to call \mods{wmconvct} in \mods{wrapcall} at the Python-level,
just type \cmd{model.\_\_qtcm.wrapcall.wmconvct()} (where \vars{model}
is a \class{Qtcm} instance).
For \mods{driver} and \mods{varptrinit}, these subroutines are not
contained in a \vars{\_\_qtcm} module, and thus can be called
directly (e.g., just type \cmd{model.\_\_qtcm.driver()}).
See Sections~\ref{sec:setbypy} and~\ref{sec:wrapcall} for more information
on the \mods{setbypy} and \mods{wrapcall} modules.

For the \vars{'full'} case, the only compiled QTCM1 model
subroutine you can usefully call during a run session is \mods{driver}.
For the \vars{'parts'} case, while you can essentially call any subroutine
given in a run list, you usually will not directly call a compiled QTCM1
model subroutine but will instead call it through including it in a
run list.  For example, if you have the following run list in a
\vars{'parts'} model:
\begin{codeblock}
\codeblockfont{%
[ 'qtcminit', '\_\_qtcm.wrapcall.woutpinit' ]}
\end{codeblock}
Running this list using the \class{Qtcm} instance method
\mods{run\_list} will result in \class{Qtcm} instance method
\mods{qtcminit} first being run, 
then the compiled QTCM1 Fortran model subroutine
\mods{woutpinit} in Fortran module \mods{wrapcall} being run.
See Section~\ref{sec:runlists} and
Table~\ref{tab:stnd.runlists} for a discussion and list of the
standard run lists that control routine execution content and order
in the \vars{'parts'} case.




%---------------------------------------------------------------------
\section{Restart and Continuation Run Sessions}
				\label{sec:contination.run.sessions}


	\subsection{Restart Runs In the Pure-Fortran QTCM1}
					\label{sec:puref90.restart}

To enable restart of a model run, the pure-Fortran QTCM1 model
writes out a restart file with the state of the prognostic variables
and select right-hand sides at that point in the run (for a list
of the variables, see Section~\ref{sec:snapshots}).  This binary
file can then be read in by later model runs.  The Fortran
\vars{mrestart} flag is passed in via a namelist; if \vars{mrestart}
is 1, the run uses the restart file (named \fn{qtcm.restart}).

One of the problems with using the restart file to do a continuation
run is that the continuation run will not be perfect.  In other words,
a 15~day run followed by a 25~day run based on the restart file 
generated at the end of the 15~day run will \emph{not} give the
exact same output as a continuous 40~day run.


	\subsection{Overview of Restart/Continuation Options In \mods{qtcm}}
					\label{sec:restart.options.list}

For a \class{Qtcm} instance, in contrast to the pure-Fortran QTCM1,
more than one method of continuation is available.
Thus, for a continuation run, you need to tell the model
``continue from what?''
The \class{Qtcm} class provides three choices for restart/continuing
a run:
\begin{enumerate} 
\item From a restart file:  Move/rename a QTCM1 restart file
        to the current working directory to \fn{qtcm.restart}.
	\label{list:continue.from.restart}

\item From a snapshot from another run session
	(see Sections~\ref{sec:snapshot.intro} and~\ref{sec:snapshots}).
	\label{list:continue.from.snapshot}

\item From the values of the \class{Qtcm} instance you will be
	calling \mods{run\_session} from.
	\label{list:continue.from.instance}
\end{enumerate}

Restart/continuation methods~\ref{list:continue.from.restart} 
and~\ref{list:continue.from.snapshot} both suffer from the
same problem as the pure-Fortran QTCM1 restart process:
They do not produce perfect restarts
(see Section~{sec:puref90.restart} for details).
In this section, I discuss the restart/continuation options
for each \vars{compiled\_form} option.

Methods~\ref{list:continue.from.restart}
and~\ref{list:continue.from.snapshot} are best used when making a
run session from a newly instantiated \class{Qtcm} instance.
Method~\ref{list:continue.from.instance} is best used when executing
a run session using a \class{Qtcm} instance that has already gone
through at least one run session.  Regardless of which method you
use, however, please note that anytime you execute a run session
using a \class{Qtcm} instance that already has made a previous run
session, some variables \emph{cannot be updated} between run sessions.
This feature is most noticeable with the output filename, and occurs
because the name persists in the compiled QTCM model, and is stored
in the extension module (\fn{.so} files in \vars{sodir}) associated
with the instance.  If you wish to control all variables possible
from the Python level (including output filename), you need do the
run session from a new model instance.


	\subsection{Restart/Continuation for 
		\vars{compiled\_form\thinspace=\thinspace'full'} 
		Model Instances}

The only option for restart when using
\vars{compiled\_form\thinspace=\thinspace'full'} model instances
is method~\ref{list:continue.from.restart}, to use a QTCM1 restart
file.\footnote%
	{The \vars{cont} keyword parameter in \mods{run\_session}
	and the value of the \vars{init\_with\_instance\_state}
	attribute have no effect if
	\vars{compiled\_form\thinspace=\thinspace'full'}.  With
	\vars{'full'}, the call to initialize variables all happens
	at the Fortran level (via the Fortran \mods{varinit}, not
	the Python \mods{varinit}), with no reference to the Python field
	states (or even existing Fortran field states, if present).}
To use this option, the value of the \vars{mrestart} 
attribute must equal 1, the restart file must be named
\fn{qtcm.restart}, and the restart file must be in the 
current working directory.
As with the pure-Fortran QTCM1 restart process, this method
does not produce perfect restarts.



	\subsection{Restart/Continuation for 
		\vars{compiled\_form\thinspace=\thinspace'parts'} 
		Model Instances}

For the \vars{compiled\_form\thinspace=\thinspace'parts'} case,
all three restart/continuation methods
described in Section~\ref{sec:restart.options.list} are
available.


		\subsubsection{Method~\ref{list:continue.from.restart}:
			From a QTCM1 Restart File}

To use the QTCM1 restart file mechanism, not only must the
\vars{mrestart} attribute have a value to 1, but the
\vars{init\_with\_instance\_state} flag also has to be \vars{False},
otherwise the \vars{mrestart} attribute value will be ignored.  
As with the pure-Fortran QTCM1 restart process, this method does not
produce perfect restarts.


		\subsubsection{Method~\ref{list:continue.from.snapshot}:
			From a \class{Qtcm} Instance Snapshot}

You can take snapshots of the model state of a \class{Qtcm} instance
by the \mods{make\_snapshot} instance method.  This snapshot saves
a copy of all the variables saved to a QTCM1 restart file (see
Section~\ref{sec:snapshots} for the full list of fields), which
then can be passed to other \class{Qtcm} instances for use in other
run sessions.

The key difference between this method and 
method~\ref{list:continue.from.instance} (described below)
is that \mods{run\_session} calls using the snapshot are done
\emph{without} the \vars{cont} keyword input parameter
(by default, \vars{cont} is False).  If the \vars{cont} keyword
is not False, it says the run session is a continuation run
that uses the state of the compiled QTCM1 model for all variables
that are not specified at, and read-in from,
the Python level.  If the \vars{cont} keyword
is False, the run session initializes as if it were a new run.

See Section~\ref{sec:snapshot.intro} for details and
an example of using snapshots to initialize a run session.
Note that as with the pure-Fortran QTCM1 restart process, this method 
does not produce perfect restarts.


		\subsubsection{Method~\ref{list:continue.from.instance}:
			From the Calling \class{Qtcm} Instance}

This method is used when you want to make a run session that is a
``true'' continuation run, i.e., one that uses the current state
of the compiled QTCM1 model for all variables that are not read-in
from the Python level (remember that \class{Qtcm} instances hold a
subset of the variables defined at the Fortran level).  
The key reason to use this method for a continuation run session
is that the continuation is byte-for-byte the same (if no fields
are changed) as if the run just went straight on through.  Thus,
the continuation would be perfect: A 15~day run followed by a 25~day
run using the same \class{Qtcm} instance with the \vars{cont} keyword
will give the exact same output as a continuous 40~day run.  This
is not the case when making a new instance and passing a restart
file or a snapshot, because a separate extension module is used for
those new instances.

Control of this method is accomplished through the \vars{cont}
keyword input parameter to the \mods{run\_session} method and the
\vars{init\_with\_instance\_state} attribute of a
\class{Qtcm} instance:

\begin{itemize}
\item \vars{cont}: If set to False, the run session is not a
	continuation of the previous run, but a new run session.
	If set to True, the run session is a continuation of the
	previous run session.  If set to an integer greater than
	zero, the run session is a continuation just like
	\vars{cont\thinspace=\thinspace{True}}, but the value
	\vars{cont} is set to is used for \vars{lastday} and replaces
	\vars{lastday.value} in the \class{Qtcm} instance.

\item \vars{init\_with\_instance\_state}:
	If True, for a \mods{run\_session} call using the
	\vars{cont} keyword, whatever the field values are in the Python
	instance are used in the run session.
	If False, model variables are set and initialized as described in
	Section~\ref{sec:init.compiledform.parts}.  In that case,
	previous compiled QTCM1 model values will likely be overwritten.
	Thus, if you want a continuation run that uses the state of
	all field variables except for those you explicitly change at
	the Python-level, make sure \vars{init\_with\_instance\_state}
	is True.
\end{itemize}

(Note that the \vars{cont} keyword has no effect if \vars{compiled\_form}
is \vars{'full'}.  The default value of \vars{cont} in a
\mods{run\_session} call is False.  The value of keyword \vars{cont}
is stored as private instance attribute \vars{\_cont}, in case you
really need to access it elsewhere; see
Section~\ref{sec:Qtcm.private.attrib} for more details).

The example described in Section~\ref{sec:continuation.intro} is
an example of method~\ref{list:continue.from.instance} in the list
above: The second run session is continued from the state of
\vars{model}, with the values of \vars{model}'s instance variables
overriding any values in the compiled QTCM1 model in initializing
the second run session.

This method has a few caveats worthy of note:
\begin{itemize}
\item The \vars{init\_with\_instance\_state} attribute value
	will have no effect unless the instance prognostic variables
	are set, i.e., unless a previous run session has been done.
	Another way to put it is for an initial run session right
	after a \class{Qtcm} instance is created, \mods{varinit}
	will use the same initial values for prognostic variables
	(defined in \mods{defaults} module variable
	\vars{init\_prognostic\_dict})\footnote%
		{\vars{init\_prognostic\_dict} is the dictionary giving
		the default initial values of each prognostic variable
		and right-hand side (as defined by the restart file 
		specification).}
	as it would with for both
	\vars{init\_with\_instance\_state} set to True or False).

\item Continuation run sessions using this method have to continue
	with the next day from wherever the last run session left
	off, contiguously.\footnote%
		{For continuation run sessions, you keep the 
		same extension module (the compiled \fn{.so} library),
		and all the values that define the state where it
		left off.}
	If you want to do a non-contiguous run,
	create a new \class{Qtcm} instance initialized with a
	snapshot instead of the continuation method describe in
	this section.
	will use restart rules to run a new model.  

\item When making a continuation run session using this method,
	you cannot change some variables, for instance,
	\vars{outdir} and any of the date related
	variables.  In fact, the only thing you should change for
	your continuation run session are the prognostic and
	diagnostic variables and \vars{lastday}.  This is because
	some variables cannot be updated between run sessions.
	As noted in Section~\ref{sec:restart.options.list},
	if you wish to control all variables possible
	from the Python level (including output filename), you need 
	to execute the run session from a new model instance.
\end{itemize}


	\subsection{Snapshots of a \class{Qtcm} Instance}
				\label{sec:snapshots}

The snapshot dictionary (briefly described in
Section~\ref{sec:snapshot.intro}), saved as the \class{Qtcm} instance
attribute \vars{snapshot}, and generated by the method
\mods{make\_snapshot}, saves the current state of the following
instance field variables:

\begin{center}
% This file is automatically generated by
% code_to_latex.py.  It lists all the snapshot variables.


\begin{longtable}{l|c|c|p{0.4\linewidth}}
\textbf{Field} & \textbf{Shape} &
                                \textbf{Units} & \textbf{Description} \\
\hline
\endhead
    \vars{T1} & (64, 44) & K &  \\
\vars{Ts} & (64, 42) & K & Surface temperature \\
\vars{WD} & (64, 42) &  &  \\
\vars{dateofmodel} &      &  & Date of model coded as an integer as yyyymmdd \\
\vars{psi0} & (64, 43) &  &  \\
\vars{q1} & (64, 44) & K &  \\
\vars{rhsu0bar} & (3,) &  &  \\
\vars{rhsvort0} & (64, 42, 3) &  &  \\
\vars{title} &      &  & A descriptive title \\
\vars{u0} & (64, 44) & m/s & Barotropic zonal wind \\
\vars{u0bar} &      &  &  \\
\vars{u1} & (64, 44) & m/s & Current time step baroclinic zonal wind \\
\vars{v0} & (64, 43) & m/s & Barotropic meridional wind \\
\vars{v1} & (64, 43) & m/s &  \\
\vars{vort0} & (64, 42) &  &  \\
\end{longtable}
\end{center}

These are the same variables saved to a QTCM1 restart file, and so
a snapshot duplicates the restart functionality in the Python
environment, but with more flexibility.  Since the \vars{snapshot}
dictionary is a Python variable like any other, you can manipulate
it and alter it to fit any condition you wish.




%---------------------------------------------------------------------
\section{Creating and Using Run Lists}  \label{sec:runlists}

Section~\ref{sec:runlist.intro} provides an introduction to the
role and use of run lists.  A run list is a list of methods, Fortran
subroutines, and other run lists that can be executed by the
\class{Qtcm} instance \mods{run\_list} method.  Run lists are stored
in the \class{Qtcm} instance attribute \vars{runlists}, which is a
dictionary of run lists.  The names of run lists should not be
preceeded by two underscores (though elements of a run list may be
very private variables), nor should names of run lists be the same
as any instance attribute.  Run lists are not available for
\vars{compiled\_form\thinspace=\thinspace'full'}.

The \mods{run\_list} method takes a single input parameter, a list,
and runs through that list of elements that specify other run lists
or instance method names to execute.  Methods with private attribute
names are automatically mangled as needed to become executable by
the method.  Note that if an item in the input run list is an
instance method, it should be the entire name (not including the
instance name) of the callable method, separated by periods as
appropriate.

Elements in a run list are either strings or 1-element dictionaries.
Consider the following example, where \vars{model} is a \class{Qtcm}
instance, and \mods{run\_list} is called using \vars{mylist} as
input:

\begin{codeblock}
\codeblockfont{%
model = Qtcm(\ldots) \\
mylist = [ \{'varinit':None\}, \\
\hspace*{13ex}'init\_model', \\
\hspace*{13ex}'\_\_qtcm.driver', \\
\hspace*{13ex}\{'set\_qtcm1\_item': ['outdir', '/home/jlin']\} ]
model.run\_list(mylist)}
\end{codeblock}

The first element in \vars{mylist} refers to a method that requires
no positional input parameters be passed in (as shown by the None).
The second and third elements in \vars{mylist} also refers to methods
that require no positional input parameters be passed in.  The last
element in \vars{mylist} refers to a method with two input parameters.
Note that while I use the term ``method'' to describe the elements,
the strings/keys do not have to be only Python instance methods.
The second element, for instance, refers to another run list, and
the third element refers to a compiled QTCM1 model subroutine (note
the \vars{\_\_qtcm} attribute).

When the \mods{run\_list} method is called, the items in the input
run list are called in the order given in the list.  For each
element,  the \mods{run\_list} method first checks if the string
or dictionary key name corresponds to the key of an entry in the
\class{Qtcm} instance attribute \vars{runlists}.  If so, \mods{run\_list}
is called using that run list (i.e., it is a ``recursive'' call).
If the string or dictionary key name does not refer to another run
list, the \mods{run\_list} method checks if the string or dictionary
key name is a method of the \class{Qtcm} instance, and if so the
method is called.  Any other value throws an exception.

If input parameters for a method are of class \class{Field}, the
\mods{run\_list} method first tries to pass the parameters into the
method as is, i.e., as Field object(s).  If that fails, the
\mods{run\_list } method  passes its parameters in as the \vars{value}
attribute of the \class{Field} object.

If you want a variable that is being passed into a run list to be
continuously updated, you have to set the parameter in the run list
to a \class{Field} instance that is a \class{Qtcm} instance attribute,
not just to the value of the field variable (or to a non-\class{Field}
object).  Otherwise, subsequent calls to that run list element will
not use the updated values as input parameters.

For instance, if you had a run list element:
\begin{codeblock}
\codeblockfont{%
\{'\_\_qtcm.timemanager':[model.coupling\_day,]\}}
\end{codeblock}
and \vars{model.coupling\_day} were an integer (it's not by default,
but pretend it was), then \mods{run\_list} calling
\mods{\_\_qtcm.timemanager} will pass in a scalar integer rather
than a binding to the variable \vars{model.coupling\_day}.  In such
a situation, if the variable \vars{model.coupling\_day} were updated
in time, the \mods{run\_list} call of \mods{\_\_qtcm.timemanager}
would not be updated in time.  This happens because when the
dictionary that is the run list element is created, the value of
list element(s) attached to the dictionary element is set to the
scalar value of \vars{model.coupling\_day} at that instant.

You can get around this feature by setting \class{Qtcm} instance
attributes that will change with model execution to \class{Field}
instances, and then referring to those attributes in the parameter
list in the run list element.  In that case:
\begin{codeblock}
\codeblockfont{%
\{'\_\_qtcm.timemanager':[model.coupling\_day,]\}}
\end{codeblock}
will use the current value of \vars{model.coupling\_day} anytime
\vars{\_\_qtcm.timemanager} is called by \mods{run\_list}, if
\vars{model.coupling\_day} is a \class{Field} object.

When \mods{run\_list}, encounters a calling input parameter that
is a \class{Field} object, it will first try to pass the entire
\class{Field} object to the method/routine being called.  If that
raises an exception, it will then try to pass just the value of the
entire \class{Field} object.  This is done to enable \mods{run\_list}
to be used for both pure-Python and compiled QTCM Fortran model
routines.  Fortran cannot handle \class{Field} objects as input
parameters, only values.

Table~\ref{tab:stnd.runlists} shows all standard run lists
stored in the \vars{runlists} attribute upon instantiation
of a \class{Qtcm} instance.

\begin{htmlonly}
\begin{table}[htp]
\begin{center}
\fbox{Empty placeholder block for table that would have gone here.}
\end{center}
\caption{Standard run lists stored in the \vars{runlists} 
	attribute upon instantiation of a \class{Qtcm} instance.
	The run list and list element names are stored as strings.
	\emphpara{This table is improperly reproduced in the
	HTML conversion.  Please see the PDF version for the table.}}
\label{tab:stnd.runlists}
\end{table}
\end{htmlonly}

\begin{latexonly}
\begin{table}[htp]
% This file is automatically generated by
% code_to_latex.py.  It lists all the standard 
% runlists in the class.


\begin{longtable}{l|lc}
\textbf{Run List Name/Description} & \textbf{List Element(s) Name(s)} & 
                                                   \textbf{\# Arg(s)} \\
\hline
\endhead
\multirow{3}{*}{\parbox{0.4\linewidth}{atm\_bartr\_mode (calculate the atmospheric barotropic mode at the barotropic timestep)}} & \_\_qtcm.wrapcall.wsavebartr & None \\
         & \_\_qtcm.wrapcall.wbartr & None \\
         & \_\_qtcm.wrapcall.wgradphis & None \\
\hline
\multirow{5}{*}{\parbox{0.4\linewidth}{atm\_oc\_step (calculate the atmosphere and ocean models at a coupling timestep)}} & \_first\_method\_at\_atm\_oc\_step & None \\
         & \_\_qtcm.wrapcall.wtimemanager & 1\\
         & \_\_qtcm.wrapcall.wocean & 2\\
         & qtcm & None \\
         & \_\_qtcm.wrapcall.woutpall & None \\
\hline
\multirow{5}{*}{\parbox{0.4\linewidth}{atm\_physics1 (calculate atmospheric physics at one instant)}} & \_\_qtcm.wrapcall.wmconvct & None \\
         & \_\_qtcm.wrapcall.wcloud & None \\
         & \_\_qtcm.wrapcall.wradsw & None \\
         & \_\_qtcm.wrapcall.wradlw & None \\
         & \_\_qtcm.wrapcall.wsflux & None \\
\hline
\multirow{8}{*}{\parbox{0.4\linewidth}{atm\_step (calculate the entire atmosphere at one atmosphere timestep)}} & atm\_physics1 & None \\
         & \_\_qtcm.wrapcall.wsland1 & None \\
         & \_\_qtcm.wrapcall.wadvctuv & None \\
         & \_\_qtcm.wrapcall.wadvcttq & None \\
         & \_\_qtcm.wrapcall.wdffus & None \\
         & \_\_qtcm.wrapcall.wbarcl & None \\
         & \_bartropic\_mode\_at\_atm\_step & None \\
         & \_\_qtcm.wrapcall.wvarmean & None \\
\hline
\multirow{3}{*}{\parbox{0.4\linewidth}{init\_model (initialize the entire model, i.e., the atmosphere and ocean components and output)}} & qtcminit & None \\
         & \_\_qtcm.wrapcall.woceaninit & None \\
         & \_\_qtcm.wrapcall.woutpinit & None \\
\hline
\multirow{5}{*}{\parbox{0.4\linewidth}{qtcminit (initialize the atmosphere portion of the entire model)}} & \_\_qtcm.wrapcall.wparinit & None \\
         & \_\_qtcm.wrapcall.wbndinit & None \\
         & varinit & None \\
         & \_\_qtcm.wrapcall.wtimemanager & 1\\
         & atm\_physics1 & None \\
\end{longtable}
\caption{Standard run lists stored in the \vars{runlists} 
	attribute upon instantiation of a \class{Qtcm} instance.
	The run list and list element names are stored as strings.}
\label{tab:stnd.runlists}
\end{table}
\end{latexonly}

Of course, feel free to change the contents of any of the run lists
after instantiation, or to add additional run lists to the
\vars{runlists} attribute dictionary.  The ability to alter run
lists at runtime gives the \mods{qtcm} package much of its flexibility.




%---------------------------------------------------------------------
\section{Field Variables and the \class{Field} Class}
						\label{sec:field.variables}

The term ``field'' variable refers to QTCM1 model variables that 
are accessible at both the compiled Fortran QTCM1 model-level as
well as the Python \class{Qtcm} instance-level.
Field variables are all instances of the \class{Field} class
(though non-field variables can also be instances of \class{Field}).

Section~\ref{sec:field.variables.intro} gives a brief introduction to
the attributes and methods in a \class{Field} instance.
A nitty gritty description of the class is found in its docstrings.

	\subsection{Creating Field Variables}

To create a \class{Field} instance whose value is set to the
default, instantiate with the field id as the only positional
input argument.  Thus:

\begin{codeblock}
\codeblockfont{foo = Field('lastday')}
\end{codeblock}

will return \vars{foo} as a \class{Field} instance with \vars{foo.value}
set to the value listed in Section~\ref{sec:defaults.scalar}.
The value of all \class{Field} instances upon creation are specified
in the \mods{defaults} submodule of package \mods{qtcm}, and listed
in Sections~\ref{sec:defaults.scalar} and~\ref{sec:defaults.array}.

To create \class{Field} instances whose attributes are set different
from their defaults, you can specify the different settings in the
instantiation parameter list, or change the attributes once the
instance is created.  See the \class{Field} docstring for details.


	\subsection{Initial Field Variables}  \label{sec:initial.variables}

Field variables include both model parameters that do not change
for a \class{Qtcm} instance as well as prognostic variables that
do change during model integration.  As a result, many field variables
have values different from the default values listed in
Sections~\ref{sec:defaults.scalar} and~\ref{sec:defaults.array}.
In this section, I list the \emph{initial} values of all field
variables.  The ``initial'' values are the settings for \class{Qtcm}
field variables execution of the \mods{run\_session} method, but
prior to cycling through an atmosphere-ocean coupling timestep.
This is in contrast to ``default'' values, which the field variables
are given on instantiation, if no other value is specified.
Numerical values are rounded as per the conventions
of Python's \vars{\%g} format code.


		\subsubsection{Scalars}

For the fields that give the input/output directory names, and the
run name, the entry ``value varies'' is provided in the ``Value''
column.

\input{init_scalars}

		\subsubsection{Arrays}

% This file is automatically generated by the script
% code_to_latex.py in the doc/latex directory.  It is based
% upon the values found after model initialization, and should
% not be hand-edited if you want the values to correspond to
% the values in a Qtcm instance, for compiled_form='parts'.


\begin{longtable}{l|c|c|c|c|p{0.3\linewidth}}
\textbf{Field} & \textbf{Shape} & \textbf{Max} & \textbf{Min} &
                                \textbf{Units} & \textbf{Description} \\
\hline
\endhead
\vars{Evap} & (64, 42) & 1502.56 & 223.552 &  &  \\
\vars{FLW} & (64, 42) & 74.5136 & 74.5136 &  &  \\
\vars{FLWds} & (64, 42) & 206.424 & 206.424 &  &  \\
\vars{FLWus} & (64, 42) & 429.708 & 429.708 &  &  \\
\vars{FLWut} & (64, 42) & 148.771 & 148.771 &  &  \\
\vars{FSW} & (64, 42) & 147.767 & 0 &  &  \\
\vars{FSWds} & (64, 42) & 410.895 & -6.99713 &  &  \\
\vars{FSWus} & (64, 42) & 356.831 & -4.49983 &  &  \\
\vars{FSWut} & (64, 42) & 332.431 & 0 &  &  \\
\vars{FTs} & (64, 42) & 930.115 & 138.383 &  &  \\
\vars{Qc} & (64, 42) & 0 & 0 & K & Precipitation \\
\vars{S0} & (64, 42) & 534.264 & 0 &  &  \\
\vars{STYPE} & (64, 42) & 3 & 0 &  & Surface type; ocean or vegetation type over land \\
\vars{T1} & (64, 44) & -100 & -100 & K &  \\
\vars{Ts} & (64, 42) & 295 & 295 & K & Surface temperature \\
\vars{WD} & (64, 42) & 350 & 0 &  &  \\
\vars{WD0} & (4,) & 500 & 0 &  & Field capacity SIB2/CSU (approximately) \\
\vars{arr1} & (64, 42) & 0 & 0 &  & Auxiliary optional output array 1 \\
\vars{arr2} & (64, 42) & 0 & 0 &  & Auxiliary optional output array 2 \\
\vars{arr3} & (64, 42) & 0.138699 & 0.138699 &  & Auxiliary optional output array 3 \\
\vars{arr4} & (64, 42) & 0 & 0 &  & Auxiliary optional output array 4 \\
\vars{arr5} & (64, 42) & 0 & 0 &  & Auxiliary optional output array 5 \\
\vars{arr6} & (64, 42) & 0 & 0 &  & Auxiliary optional output array 6 \\
\vars{arr7} & (64, 42) & 0 & 0 &  & Auxiliary optional output array 7 \\
\vars{arr8} & (64, 42) & 0 & 0 &  & Auxiliary optional output array 8 \\
\vars{psi0} & (64, 43) & 0 & 0 &  &  \\
\vars{q1} & (64, 44) & -50 & -50 & K &  \\
\vars{rhsu0bar} & (3,) & 0 & 0 &  &  \\
\vars{rhsvort0} & (64, 42, 3) & 0 & 0 &  &  \\
\vars{taux} & (64, 42) & 0 & 0 &  &  \\
\vars{tauy} & (64, 42) & 0 & 0 &  &  \\
\vars{u0} & (64, 44) & 0 & 0 & m/s & Barotropic zonal wind \\
\vars{u1} & (64, 44) & 0 & 0 & m/s & Current time step baroclinic zonal wind \\
\vars{v0} & (64, 43) & 0 & 0 & m/s & Barotropic meridional wind \\
\vars{v1} & (64, 43) & 0 & 0 & m/s &  \\
\vars{vort0} & (64, 42) & 0 & 0 &  &  \\
\end{longtable}



	\subsection{Passing Fields Between the Python and Fortran-Levels}

Section~\ref{sec:comm.py.fort.compiledform} discusses the differences
between how the \vars{'full'} and \vars{'parts'} compiled forms
pass field variables between the Python and Fortran-levels.  That
discussion gives a detailed description of the methods used for
passing fields to and from the Python and Fortran-levels (i.e., the
\mods{get\_qtcm1\_item} and \mods{set\_qtcm1\_item} methods).

Please note the following regarding field variables as you pass them 
back and forth between the Python and Fortran-levels:
\begin{itemize}
\item Field variables with ghost latitudes, such as \vars{u1}, on
	the Python end are always the full variables (i.e., including
	the ghost latitudes).  On the Fortran end, variables like
	\vars{u1} also always have the ghost latitudes while in the
	model, but when stored as restart files, do not have the
	ghost latitudes; the end points are not saved in restart
	files or written to the netCDF output files.
	See the
	\latexhtml{%
\htmladdnormallinkfoot{QTCM1 manual}%
        {http://www.atmos.ucla.edu/$\sim$csi/qtcm\_man/v2.3/qtcm\_manv2.3.pdf}}%
{\htmladdnormallink{QTCM1 manual}%
        {http://www.atmos.ucla.edu/~csi/qtcm_man/v2.3/qtcm_manv2.3.pdf}}
	\cite{Neelin/etal:2002}
	for details about ghost latitudes.

\item You should assume there is only a full synchronizing between 
	compiled QTCM1 model and Python model field variables
	at the beginning and end of a run session.  

\item If you have a variable at the Python-level, but at the
	compiled QTCM1 Fortran model-level the variable is not
	readable, if you try to call \mods{set\_qtcm1\_item} on the
	variable, nothing is done, and the Python-level value is
	left alone.  If you have a compiled QTCM1 model variable,
	but no Python-level equivalent, if you call \mods{set\_qtcm1\_item}
	on the variable, the Python-level variable (as an attribute)
	is created.

\item To be precise, only compiled QTCM1 model variables can be
	passed pass back and forth between the Python and Fortran-levels;
	there are many \class{Qtcm} instance attributes that do not
	have any counterparts at the Fortran-level.\footnote%
		{I use the term ``field variables'' to refer to 
		compiled QTCM1 model variables that can be passed
		back and forth to the Python level.}

\item Although \vars{dayofmodel} is described in module \mods{setbypy}
	as an option for the \mods{get\_qtcm1\_item} and
	\mods{set\_qtcm1\_item} methods to operate on, in reality
	those methods cannot operate on \vars{dayofmodel}, but
	\vars{dayofmodel} is not defined in \mods{defaults}.\footnote%
		{All field variables must be defined in \mods{defaults} in
		order for the proper Fortran routine to be called
		according to the variable's type.}
\end{itemize}


	\subsection{Field Variable Shape}   \label{sec:field.var.shape}

Normally, Python arrays have a different dimension order than Fortran
arrays.  While Fortran arrays are dimensioned (col, row, slice),
with adjacent columns being contiguous, then rows, and then slices, Python
arrays are dimensioned (slice, row, col), with adjacent columns being
contiguous, then rows, and then slices.  Based on this, you would
think that everytime you passed an array between the Python and
Fortran-levels you would need to transpose the array.

Thankfully, we don't have to do this because \mods{f2py} handles
array dimension order transparently so we can refer to each element
the same way whether we're in Python or Fortran.  Thus, the array
\vars{Qc} in Fortran is dimensioned (longitude, latitude), (64,42)
by default, and the Python \class{Qtcm} instance attribute \vars{Qc}
has a \vars{value} attribute also dimensioned (longitude, latitude),
(64,42) by default.  And at both the Fortran and Python-levels, the
first longtude, second latitude element is referred to as \vars{Qc(1,2)}.

In contrast, however, netCDF output saved by the compiled QTCM1 model
and read into Python (using the \mods{Scientific} package) is
\emph{not} in Fortran array order.  Arrays read from netCDF output
into Python are in Python array order, and are dimensioned
(latitude, longitude) or (time, latitude, longitude).  The \class{Qtcm}
routines that manipulate netCDF data (e.g., \mods{plotm}), however,
automatically adjust for this, so you only need to be aware of this
when reading in output for your own analysis
(see Section~\ref{sec:model.output}).




%---------------------------------------------------------------------
\section{Model Output}			\label{sec:model.output}

Section~\ref{sec:output.intro} gives an overview of how to
use \mods{qtcm} model output to netCDF files.

All netCDF array output is dimensioned (time, latitude, longitude)
when read into Python using the \mods{Scientific} package.  This
differs from the way \class{Qtcm} saves field variables, which
follows Fortran convention (longitude, latitude).  Thus, the shapes
in Section~\ref{sec:initial.variables}, Appendix~\ref{app:defaults.values},
etc., are not the shapes of arrays read from the netCDF output.
See Section~\ref{sec:field.var.shape} for a discussion of why
there is this discrepancy.

Because netCDF files allow you to specify an ``unlimited'' dimension,
it is possible to close a netCDF file, reopen it, and add more
slices of data to the file.  Thus, continuous \class{Qtcm} run
sessions (i.e., those that use the \vars{cont} keyword input parameter
in the \mods{run\_session} method) will automatically append output
to the netCDF output files.

Field variables with ghost latitudes, such as \vars{u1}, on the
Python and Fortran ends are always the full variables (i.e., including
the ghost latitudes).  The ghost latitudes are not written to the
netCDF output files, however.
See the \latexhtml{%
\htmladdnormallinkfoot{QTCM1 manual}%
        {http://www.atmos.ucla.edu/$\sim$csi/qtcm\_man/v2.3/qtcm\_manv2.3.pdf}}%
{\htmladdnormallink{QTCM1 manual}%
        {http://www.atmos.ucla.edu/~csi/qtcm_man/v2.3/qtcm_manv2.3.pdf}}
	\cite{Neelin/etal:2002}
for details about ghost latitude structure.

\class{Qtcm} instances have a few built-in tools to visualization
model output.  These are briefly described in Section~\ref{sec:viz.intro}.
Note that the \mods{plotm} method is linked to a specific \class{Qtcm}
instance.  Do not use \mods{plotm} outside of the instance it is
linked to.  It must also be used only after a successful run session
(i.e., not in the middle of a run session).




%---------------------------------------------------------------------
\section{Miscellaneous}

A few miscellaneous items/issues about the model:
\begin{itemize}
\item The land model runs at same timestep as the atmosphere.

\item If the land model runs less often than 
	\mods{sflux} in \mods{physics1}, 
	the calculation of evaporation over the land 
	needs to be fixed in sflux.

\item The units of some field variables are not what you would expect.
	For instance, \vars{Qc} is in energy units, i.e., K, and not
	mm/day.
	See the
	\latexhtml{%
\htmladdnormallinkfoot{QTCM1 manual}%
        {http://www.atmos.ucla.edu/$\sim$csi/qtcm\_man/v2.3/qtcm\_manv2.3.pdf}}%
{\htmladdnormallink{QTCM1 manual}%
        {http://www.atmos.ucla.edu/~csi/qtcm_man/v2.3/qtcm_manv2.3.pdf}}
	\cite{Neelin/etal:2002}
	for details.
\end{itemize}




%---------------------------------------------------------------------
\section{Cookbook of Ways the Model Can Be Used}  \label{sec:cookbook}

This cookbook of a few ways to use the model is arranged by science
tasks, i.e., certain types of runs we want to do.  For some of the
examples below, I assume that the dictionary
\vars{inputs} is initially defined as given in
Figure~\ref{fig:defn.of.inputs}.  All examples assume that
\cmd{from qtcm import Qtcm} has already been executed.


%--- Two versions, one for PDF and the other for HTML:
\begin{latexonly}
\begin{figure}[tp]
\begin{codeblock}
\codeblockfont{%
inputs = \{\} \\
inputs['runname'] = 'test' \\
inputs['landon'] = 0 \\
inputs['year0'] = 1 \\
inputs['month0'] = 11 \\
inputs['day0'] = 1 \\
inputs['lastday'] = 30 \\
inputs['mrestart'] = 0 \\
inputs['init\_with\_instance\_state'] = True \\
inputs['compiled\_form'] = 'parts'}
\end{codeblock}

\caption{The initial definition of the \vars{inputs} dictionary for 
	examples given in Section~\ref{sec:cookbook}.  These settings
	imply that a run session will start on November 1, Year 1,
	last for 30 days, and will be an aquaplanet run.}
\label{fig:defn.of.inputs}
\end{figure}
\end{latexonly}

\begin{htmlonly}
\label{fig:defn.of.inputs}
\begin{center}
\htmlfigcaption{%
	\codeblockfont{%
inputs = \{\} \\
inputs['runname'] = 'test' \\
inputs['landon'] = 0 \\
inputs['year0'] = 1 \\
inputs['month0'] = 11 \\
inputs['day0'] = 1 \\
inputs['lastday'] = 30 \\
inputs['mrestart'] = 0 \\
inputs['init\_with\_instance\_state'] = True \\
inputs['compiled\_form'] = 'parts'}
	}

\htmlfigcaption{Figure~\ref{fig:defn.of.inputs}:
	The initial definition of the \vars{inputs} dictionary for 
	examples given in Section~\ref{sec:cookbook}.  These settings
	imply that a run session will start on November 1, Year 1,
	last for 30 days, and will be an aquaplanet run.}
\end{center}
\end{htmlonly}



\begin{description}
\item[Plain model run:]
	Here I just want to make a single model run.
	Tasks:  Instantiate a fresh model and execute a run session.
	The code to run the model is just:
	\begin{codeblock}
	\codeblockfont{%
inputs['init\_with\_instance\_state'] = False \\
model = Qtcm(**inputs) \\
model.run\_session()}
	\end{codeblock}
	where \vars{inputs} is initialized with the code in
	Figure~\ref{fig:defn.of.inputs}.


\item[Explore parameter space with a set of models:]
	Here I want to create an entire suite of separate models,
	in order to determine the sensitivity of the model to changing
	a parameter.
	To do this, I
	instantiate multiple fresh models, 
	and execute a run session for each instance, all within
	a \vars{for} loop:


%--- Two versions, because LaTeX2HTML does not correctly typeset
%    the hspace command:
\begin{latexonly}
	\begin{codeblock}
	\codeblockfont{%
import os \\
inputs['init\_with\_instance\_state'] = False \\
for i in xrange(0,1002,10): \\
\hspace*{5ex}iname = 'ziml-' + str(i) + 'm' \\
\hspace*{5ex}ipath = os.path.join('proc', iname) \\
\hspace*{5ex}os.makedirs(ipath) \\
\hspace*{5ex}model = Qtcm(**inputs) \\
\hspace*{5ex}model.ziml.value = float(i)  \\
\hspace*{5ex}model.runname.value = iname \\
\hspace*{5ex}model.outdir.value = ipath \\
\hspace*{5ex}model.run\_session() \\
\hspace*{5ex}del model}
	\end{codeblock}
\end{latexonly}

\begin{htmlonly}
\begin{center}
\htmlfigcaption{%
	\codeblockfont{%
import os \\
inputs['init\_with\_instance\_state'] = False \\
for i in xrange(0,1002,10): \\
\hspace*{5ex}iname = 'ziml-' + str(i) + 'm' \\
\hspace*{5ex}ipath = os.path.join('proc', iname) \\
\hspace*{5ex}os.makedirs(ipath) \\
\hspace*{5ex}model = Qtcm(**inputs) \\
\hspace*{5ex}model.ziml.value = float(i)  \\
\hspace*{5ex}model.runname.value = iname \\
\hspace*{5ex}model.outdir.value = ipath \\
\hspace*{5ex}model.run\_session() \\
\hspace*{5ex}del model}
	}
\end{center}
\end{htmlonly}


	The loop explores mixed-layer depth \vars{ziml} from 0~m to
        1000~m, in 10~m intervals.  I create the \vars{outdir}
	directory before every model call, since the compiled QTCM1 model
	requires the output directory exist, specifying the run name
	and output directory as the string \vars{iname}.
	The output directories are assumed to all be in the \fn{proc}
	sub-directory of the current working directory.
	\vars{inputs} is initialized with the code in
	Figure~\ref{fig:defn.of.inputs}.


\item[Conditionally explore parameter space:]
	Here I want to 
	conditionally explore the parameter space, on the basis of
	some mathematical criteria.
	To do this, I
	instantiate a model, evaluate results using
	that criteria, and run another fresh model depending on
	the results (passing the previous model state via a snapshot),
	all within a \vars{while} loop.
	Note that this type of investigation is very difficult to 
	automate if all you can use are shell scripts and
	Fortran.
	See Figure~\ref{fig:conditional.test.eg} for a detailed
	example.


\item[With interactive adjustments at run time:]
	The example in Figure~\ref{sec:continuation.intro}
	illustrates this type of run.  In this example,
	I instantiate a fresh model, execute a run session, analyze the
	output, change variables in the model instance, and then
	execute a continuation run session.


\item[Test alternative parameterizations:]
	I've already described how we can use run lists to arbitrarily
	change model execution order and content at run time.
	We can take advantage of Python's inheritance
	abilities, along with run lists, to simplify this.
	Figure~\ref{fig:alt.param.inherit.eg} provides an example of
	this use.

	Of course, you can use pre-processor directives and shell
	scripts to accomplish the same functionality seen in
	Figure~\ref{fig:alt.param.inherit.eg} using just Fortran.
	The Python solution, however, shortcuts the compile/linking
	step, and enables you to easily do run time swapping between
	subroutine choices based upon run time calculated
	tests (see Figure~\ref{fig:conditional.test.eg} for an
	example of such tests).
\end{description}




% --- Two versions of this block, one for display in PDF and the other
%     for display in HTML:
\begin{latexonly}
\begin{figure}[p]
	\begin{codeblock}
	\codeblockfont{%
\small
import os \\
import numpy as N \\
maxu1 = 0.0 \\
while maxu1 < 10.0: \\
\hspace*{5ex}iziml = 0.1 * maxu1 \\
\hspace*{5ex}iname = 'ziml-' + str(iziml) + 'm' \\
\hspace*{5ex}ipath = os.path.join('proc', iname) \\
\hspace*{5ex}os.makedirs(ipath) \\
\hspace*{5ex}model = Qtcm(**inputs) \\
\hspace*{5ex}try: \\
\hspace*{10ex}model.sync\_set\_py\_values\_to\_snapshot(snapshot=mysnapshot) \\
\hspace*{10ex}model.init\_with\_instance\_state = True \\
\hspace*{5ex}except: \\
\hspace*{10ex}model.init\_with\_instance\_state = False \\
\hspace*{5ex}model.ziml.value = iziml  \\
\hspace*{5ex}model.runname.value = iname \\
\hspace*{5ex}model.outdir.value = ipath \\
\hspace*{5ex}model.run\_session() \\
\hspace*{5ex}maxu1 = N.max(N.abs(model.u1.value)) \\
\hspace*{5ex}mysnapshot = model.snapshot \\
\hspace*{5ex}del model}
	\end{codeblock}

\caption{This code explores different values of
	mixed-layer depth \vars{ziml} for 30~day runs,
	as a function of maximum \vars{u1} magnitude,
	until it finds a case where the maximum \vars{u1} is
	greater than 10~m/s.  (The relationship between
	\vars{ziml} and the maximum of the speed of
	\vars{u1}, where 
	\vars{ziml\thinspace=\thinspace0.1\thinspace*\thinspace{maxu1}}, 
	is made up.)
	With each iteration, the new run uses the snapshot from
	a previous run to initialize (as well as the new value
	of \vars{ziml}); the \vars{try} statement is used to
	ensure the model works even if \vars{mysnapshot} is not
	defined (which is the case the first time around).
	The \vars{inputs} dictionary is initialized with the code in
	Figure~\ref{fig:defn.of.inputs}.}
\label{fig:conditional.test.eg}
\end{figure}
\end{latexonly}

\begin{htmlonly}
\label{fig:conditional.test.eg}
\begin{center}
\htmlfigcaption{%
	\codeblockfont{%
import os \\
import numpy as N \\
maxu1 = 0.0 \\
while maxu1 < 10.0: \\
\hspace*{5ex}iziml = 0.1 * maxu1 \\
\hspace*{5ex}iname = 'ziml-' + str(iziml) + 'm' \\
\hspace*{5ex}ipath = os.path.join('proc', iname) \\
\hspace*{5ex}os.makedirs(ipath) \\
\hspace*{5ex}model = Qtcm(**inputs) \\
\hspace*{5ex}try: \\
\hspace*{10ex}model.sync\_set\_py\_values\_to\_snapshot(snapshot=mysnapshot) \\
\hspace*{10ex}model.init\_with\_instance\_state = True \\
\hspace*{5ex}except: \\
\hspace*{10ex}model.init\_with\_instance\_state = False \\
\hspace*{5ex}model.ziml.value = iziml  \\
\hspace*{5ex}model.runname.value = iname \\
\hspace*{5ex}model.outdir.value = ipath \\
\hspace*{5ex}model.run\_session() \\
\hspace*{5ex}maxu1 = N.max(N.abs(model.u1.value)) \\
\hspace*{5ex}mysnapshot = model.snapshot \\
\hspace*{5ex}del model}
	}

\htmlfigcaption{Figure \ref{fig:conditional.test.eg}:
	This code explores different values of
	mixed-layer depth \vars{ziml} for 30~day runs,
	as a function of maximum \vars{u1} magnitude,
	until it finds a case where the maximum \vars{u1} is
	greater than 10~m/s.  (The relationship between
	\vars{ziml} and the maximum of the speed of
	\vars{u1}, where 
	\vars{ziml\thinspace=\thinspace0.1\thinspace*\thinspace{maxu1}}, 
	is made up.)
	With each iteration, the new run uses the snapshot from
	a previous run to initialize (as well as the new value
	of \vars{ziml}); the \vars{try} statement is used to
	ensure the model works even if \vars{mysnapshot} is not
	defined (which is the case the first time around).
	The \vars{inputs} dictionary is initialized with the code in
	Figure~\ref{fig:defn.of.inputs}.}
\end{center}
\end{htmlonly}


% --- Two versions of this block, one for display in PDF and the other
%     for display in HTML:
\begin{latexonly}
\begin{figure}[p]
\begin{center}
	\begin{codeblock}
	\codeblockfont{%
\small
import os \\
\\
class NewQtcm(Qtcm): \\
\hspace*{5ex}def cloud0(self):\\
\hspace*{10ex}[\ldots] \\
\hspace*{5ex}def cloud1(self):\\
\hspace*{10ex}[\ldots] \\
\hspace*{5ex}def cloud2(self):\\
\hspace*{10ex}[\ldots] \\
\hspace*{5ex}[\ldots] \\
\\
inputs['init\_with\_instance\_state'] = False \\
for i in xrange(10): \\
\hspace*{5ex}iname = 'cloudroutine-' + str(i)  \\
\hspace*{5ex}ipath = os.path.join('proc', iname) \\
\hspace*{5ex}os.makedirs(ipath) \\
\hspace*{5ex}model = NewQtcm(**inputs) \\
\hspace*{5ex}model.runlists['atm\_physics1'][1] = 'cloud' + str(i) \\
\hspace*{5ex}model.runname.value = iname \\
\hspace*{5ex}model.outdir.value = ipath \\
\hspace*{5ex}model.run\_session() \\
\hspace*{5ex}del model}
	\end{codeblock}
\end{center}

\caption{Let's say we have 9 different cloud physics schemes we wish
	to try out in 9 different runs.  The easiest way to do this
	is to create a new class \class{NewQtcm} that
	inherits everything from \class{Qtcm}, and to which we'll
	add the additional cloud schemes (\vars{cloud0}, \vars{cloud1},
	etc.).
	In the \vars{for} loop, I change the cloud model
	run list entry in the run list that governs
	atmospheric physics at one instant to whatever the cloud
	model is at this point in the loop.
	The \vars{inputs} dictionary is initialized with the code in
	Figure~\ref{fig:defn.of.inputs}.
	Of course, we could do the same thing by running the 9
	models separately, but this set-up makes it easy to do
	hypothesis testing with these 9 models.  For instance, we
	can create a test by which we will choose which of the 9
	models to use:  Within this framework, the selection of
	those models can be altered by changing a string.}
\label{fig:alt.param.inherit.eg}
\end{figure}
\end{latexonly}

\begin{htmlonly}
\label{fig:alt.param.inherit.eg}
\begin{center}
\htmlfigcaption{%
	\codeblockfont{%
import os \\
\\
class NewQtcm(Qtcm): \\
\hspace*{5ex}def cloud0(self):\\
\hspace*{10ex}[\ldots] \\
\hspace*{5ex}def cloud1(self):\\
\hspace*{10ex}[\ldots] \\
\hspace*{5ex}def cloud2(self):\\
\hspace*{10ex}[\ldots] \\
\hspace*{5ex}[\ldots] \\
\\
inputs['init\_with\_instance\_state'] = False \\
for i in xrange(10): \\
\hspace*{5ex}iname = 'cloudroutine-' + str(i)  \\
\hspace*{5ex}ipath = os.path.join('proc', iname) \\
\hspace*{5ex}os.makedirs(ipath) \\
\hspace*{5ex}model = NewQtcm(**inputs) \\
\hspace*{5ex}model.runlists['atm\_physics1'][1] = 'cloud' + str(i) \\
\hspace*{5ex}model.runname.value = iname \\
\hspace*{5ex}model.outdir.value = ipath \\
\hspace*{5ex}model.run\_session() \\
\hspace*{5ex}del model}
	}

\htmlfigcaption{Figure \ref{fig:alt.param.inherit.eg}:
	Let's say we have 9 different cloud physics schemes we wish
	to try out in 9 different runs.  The easiest way to do this
	is to create a new class \class{NewQtcm} that
	inherits everything from \class{Qtcm}, and to which we'll
	add the additional cloud schemes (\vars{cloud0}, \vars{cloud1},
	etc.).
	In the \vars{for} loop, I change the cloud model
	run list entry in the run list that governs
	atmospheric physics at one instant to whatever the cloud
	model is at this point in the loop.
	The \vars{inputs} dictionary is initialized with the code in
	Figure~\ref{fig:defn.of.inputs}.
	Of course, we could do the same thing by running the 9
	models separately, but this set-up makes it easy to do
	hypothesis testing with these 9 models.  For instance, we
	can create a test by which we will choose which of the 9
	models to use:  Within this framework, the selection of
	those models can be altered by changing a string.}
\end{center}
\end{htmlonly}




% ===== end of file =====
