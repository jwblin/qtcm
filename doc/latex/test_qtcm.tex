% ==========================================================================
% Installation Summary
%
% By Johnny Lin
% ==========================================================================


% ------ BODY -----
%

The \mods{qtcm} distribution comes with a set of tests for the
package, using Python's \mods{unittest} package.  
Just to warn you, the tests take around an hour to run.
The tests will not work if the contents of \fn{lib}
after you've finished building \mods{qtcm} have not been copied
to a directory named \fn{qtcm} that is on your \mods{sys.path} path,
so make sure you've gone through all the install steps
(summarized in Section~\ref{sec:install.sum}) before you do these
tests.

\emphpara{NB:}  For these tests to work, both \cmd{python} and
\cmd{python2.4} must refer to the executable for the Python
installation on your system that you are using for running \mods{qtcm}.

The tests require a set of benchmark output files in the
\fn{test/benchmarks} directory in the
\fn{qtcm-0.1.2}directory (the output will be in
directories whose names begin with ``aquaplanet'' or ``landon'').
These output files are not included with the \mods{qtcm} distribution,
and must be created, by doing the following:

\begin{enumerate}
\item Goto the directory \fn{test/benchmarks/create/src} in the
	\fn{qtcm-0.1.2}\mods{qtcm} distribution directory,
	and copy the makefile from sub-directory \fn{Makesfiles},
	that corresponds to your system to the
	\fn{test/benchmarks/create/src} directory.  Rename the makefile 
	in \fn{test/benchmarks/create/src} to \fn{makefile}.

\item In \fn{makefile}, make the following changes:
        \begin{enumerate}
        \item Change the \vars{FC} environment variable as needed,
                if your Fortran compiler is different.
        \item Change the \vars{FFLAGSM} environment variable, if the
                compiler flags listed are not supported by your
                compiler.
        \item Change the \vars{-I} and \vars{-L} parts of the
                \vars{NCINC} and \vars{NCLIB} environment
                variables so that the paths for the netCDF library and
                include files match your system's installation:
                \begin{codeblock}
                \codeblockfont{%
NCINC=-I/yourpath/netcdf/include \\
NCLIB=-L/yourpath/netcdf/lib -lnetcdf}
                \end{codeblock}
                Set \dumarg{yourpath} to the full path to the
                \fn{netcdf} directory where the \fn{include} and
                \fn{lib} sub-directories are that hold the netCDF
                libraries and include files.
                (You shouldn't have to change the \vars{-l} part of
                \vars{NCLIB}, since it is standard to name the netCDF
                library \fn{libnetcdf.a}.  But if you have a non-standard
                installation, change the \vars{-l} part too.)
        \end{enumerate}

\item Go to the directory \fn{test/benchmarks/create} in the
	\fn{qtcm-0.1.2}\mods{qtcm} distribution directory.

\item Type \cmd{python create\_benchmarks.py} at the Unix command line
	to run the benchmark creation script.
\end{enumerate}

The created benchmarks will be located in 
\fn{test/benchmarks}, in directories with names related to the
run that was done, as described earlier.
The benchmarks are created using the
pure-Fortran QTCM1 model code,
version 2.3 (August 2002), with an altered makefile
(described above) and the following code change:
In all \fn{.F90} files, occurrences of:
        \begin{codeblock}
        \codeblockfont{%
        Character(len=130)}
        \end{codeblock}
        are changed to:
        \begin{codeblock}
        \codeblockfont{%
        Character(len=305)}
        \end{codeblock}
This enables the model to properly deal with longer filenames.
The number ``305'' is chosen to make search and replace easier.

Once the benchmarks are created, you can test the \mods{qtcm} package
by doing the following:
\begin{enumerate}
\item Go to the \fn{test} directory in the 
	\fn{qtcm-0.1.2}directory.
\item Type \cmd{python test\_all.py} at the Unix command line.
\end{enumerate}

If at the end of the test runs you see this message (or something similar):
\begin{codeblock}
\codeblockfont{%
\footnotesize
---------------------------------------------------------------------- \\
Ran 93 tests in 1244.205s \\
 \\
OK}
\end{codeblock}
then everything worked fine!  If you get any other message, the test(s) have
failed.



% ===== end of file =====
