% ==========================================================================
% Future
%
% By Johnny Lin
% ==========================================================================


% ------ BODY -----
%
This section describes the features and fixes I plan to work on
in this package.  The most urgent items are listed closer to the
begining of the lists.

\begin{itemize}
\item Add \code{implicit none} top setbypy.F90.

\item Check through Fortran routines that have arguments, to make sure
	f2py is properly understanding the intentions
	(i.e., in, out, inout) of the variables, since we're using the
	``quick way'' of making shared object libraries using f2py.
	The \fn{utilities.F90} file has a number of Fortran routines
	with arguments.

\item Cite:  Peterson, P. (2009) 
	F2PY: a tool for connecting Fortran and Python programs, 
	\emph{Int. J. Computational Science and Engineering,}
	Vol.\ 4, No.\ 4, pp.\ 296--305 for f2py.

\item Create a method like \mods{calc\_derived('T100')} which would
	primarily operate on a data file and provide a derived variable
	such as the temperature at 100 hPa, as given in this example.
	Figure out where to put the parameters (V1s, etc.) that are
	needed to make such a calculation.  As attributes?  Create a
	method to write the quantity out to an output file?
	Perhaps make an ability to calculate these values at heights
	at a given time each day during a run session?

\item Automate the installation using Python's
\htmladdnormallinkfoot{\mods{distutils}}{http://docs.python.org/dist/dist.html}
	utilities.

\item Describe a way of using job control (either via the operating system
	or IPython's \mods{jobctrl} module) 
	to do a quick-and-dirty parallelization of multiple
	\class{Qtcm} instance run sessions.  Or use some sort of threading
	to fire up two simulataneously running models.  Check that the
	simultaneously running models have different memory space.

\item Add capability for \fn{create\_benchmark.py} to overwrite
	existing benchmark files.

\item Make \vars{compiled\_form} set to \vars{'parts'} as the
	default instantiation.  Change documentation accordingly.

\item Currently, the \class{Qtcm} \mods{plotm} method works only on
	3-D output (time, latitude, longitude).  Some of the fields
	in the netCDF output files are 2-D.  Add the capability to
	\mods{plot\_netcdf\_output} in the \mods{plot} submodule
	to handle 2-D fields.

\item Add documentation about removing temporary files.
	Add documentation in Section~\ref{sec:model.instances}
	of details of what occurs during instantiation of 
	a \class{Qtcm} instance.

\item Add the units and long names for all field variables in the
	\mods{defaults} module.

\item Create a keyword to automatically change precipitation and
	evaporation units to mm/day (or similar).

\item Add ability to calculate and plot fields at different pressure
	levels.  Create another module like defaults that specifies
	the vertical fields and gives the equation to use to calculate
	those fields; call the module ``derivfields'' or something
	similar.

\item Throughout the \mods{qtcm} package I use the condition
	\mods{N.rank(}\dumarg{arg}\mods{)\thinspace=\thinspace0} 
	to test whether
	\dumarg{arg} is a scalar.  This works fine for \mods{numpy}
	objects, but it does not work properly for
	\mods{Numeric} and \mods{numarray} arrays.  In those
	array packages, \mods{rank('abc')} returns the value~1.
	This is not a problem, as long as everyone has \mods{numpy},
	but in order to make the package interoperable, I need to
	find a better way of testing for scalars.  The definitions
	of isscalar need to be changed in \mods{num\_settings}.

\item \mods{num\_settings} needs to be changed to truly enable me
	to test whether \mods{qtcm} works for 
	\mods{numarray} and \mods{Numeric} arrays.  The tests
	do not do this right now, because \mods{num\_settings}
	defaults to \mods{numpy}, if it exists.

\item Create makefiles for other platforms.
 
\item A few fields (e.g., \vars{u1}) have data for extra latitude bands,
	due to the use of ``ghost latitudes'' as part of the
	implementation of the numerics.  Details are found in the 
\latexhtml{%
\htmladdnormallinkfoot{QTCM1 manual}%
        {http://www.atmos.ucla.edu/$\sim$csi/qtcm\_man/v2.3/qtcm\_manv2.3.pdf}}%
{\htmladdnormallink{QTCM1 manual}%
        {http://www.atmos.ucla.edu/~csi/qtcm_man/v2.3/qtcm_manv2.3.pdf}}
\cite{Neelin/etal:2002}.

	Though adjusting to this idiosyncracy is not that difficult, 
	in the future I hope to implement a method of handing
	fields with ghost latitudes so that they have the same
	dimensions as the other gridded output variables.  In order
	to do this, I plan to write a Python method to read the
	Fortran generated binary restart file.

\item Change the \mods{set\_qtcm\_item} method so that it can 
	automatically accomodate setting Fortran real variables
	if integer values are input.

\item Currently, the \mods{get\_item\_qtcm} and 
	\mods{set\_item\_qtcm} methods will not work
	on integer and character arrays, only scalars and real arrays.
	Add that missing functionality to those methods.

\item Currently, the \mods{make\_snapshot} method duplicates the
	functionality of the pure-Fortran QTCM1 restart file mechanism.
	However, the restart file mechanism itself does not do a true
	restart.  A continuous run does not provide the same results
	as two runs over the same period, joined by the restart file.

	To see whether saving more variables would do the trick,
	I altered \mods{make\_snapshot} to store all Python level
	variables (i.e., \vars{self.\_qtcm\_fields\_ids}).  However,
	the restart failing described above still continued.  In the
	future, I hope to figure out exactly how many variables are
	needed in order to make the restart feature do a true
	restart.

\item Add a test of using the \vars{mrestart\thinspace=\thinspace1}
	restart option.  Does the \fn{qtcm.restart} file need to be
	in the current working directory or another?

\item Add a test in the unit test scripts to
	confirm that the \vars{init\_with\_instance\_state}
	attribute setting only has an effect if 
	\vars{compiled\_form\thinspace=\thinspace'parts'}.

\item Document \vars{tmppreview} keyword in \mods{plot.plot\_ncdf\_output}.

\item Confirm and document that
	for netCDF output, time is model time since dd-mm-yyyy.

\item Add to the \mods{plotm} method the ability to
	plot as text onto the figure the
	runname string and the calling line
	for the plotm method.

\item Couple with the
	\latexhtml{CliMT\footnote{http://maths.ucd.ie/$\sim$rca/climt/}}%
	{\htmladdnormallink{CliMT}{http://maths.ucd.ie/~rca/climt/}}
	climate modeling toolkit.

\item Enable Python to set \vars{arr1name}, etc., which are string
	variables at the Python level.  I haven't really thought through
	how \vars{arr1} variables work with the Python \class{Qtcm}
	instance.

\item Possible:  In the \class{Qtcm} method
	\mods{\_\_setattr\_\_}, add a test to raise an exception
	if the instance tries to set \vars{viscU}, \vars{viscT},
	or \vars{viscQ} as attributes.  Also create a method
	\code{isotropic\_visc} that will set all viscosity parameters
	non-dependent on direction.  See Section~\ref{sec:driverinit.diffs}
	for details.

\item Go through the manual and create HTML-only versions of tables
	that have table numbers (use a similar construct as in
	figure environments).

\item Go through documentation to check that
	output variable names are capitalized consistently.

\item Create way to redirect stdout.

\item Create a step method to run an arbitrary number of timesteps at
	the atmosphere level.

\end{itemize}


% ===== end of file =====
