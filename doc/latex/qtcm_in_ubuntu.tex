% ==========================================================================
% Description of installing in Ubuntu
%
% By Johnny Lin
% ==========================================================================


% ------ BODY -----
%
%------------------------------------------------------------------------
\subsection{Introduction}

This section describes installation issues 
I followed to install \mods{qtcm} on my
Dell PowerEdge 860 running Ubuntu GNU/Linux 8.04.1 LTS (Hardy).
The machine has 2.66 GHz Quad Core Intel Xeon processors (64 bit),
4 GB RAM, and 677 GB of disk in its main partition.
This section is a specific realization of the general installation
instructions found in Sections~\ref{sec:install.sum}--\ref{sec:test.qtcm}.
I worked through these installation steps during July 2008.
The best way to go through this section is to go through
the summary of the installation steps in 
Section~\ref{sec:ubuntu.install.summary},
and looking back to other sections as needed.



%------------------------------------------------------------------------
\subsection{Fortran Compiler}     \label{sec:ubuntu.fort.install}

The easiest Fortran compiler to install in Ubuntu 8.04.1 is
\htmladdnormallinkfoot{GNU Fortran}{http://gcc.gnu.org/fortran/}
(\mods{gfortran}), the Fortran 95 compiler included with the more
recent versions of the GNU Compiler Collection (GCC); you can
use any package manager (e.g., \mods{apt-get}, \mods{aptitude})
to install it.
Unfortunately, I was not able to get it to compile QTCM1.
I was, however, able to successfully compile QTCM1 using
the second open-source Fortran compiler,
\htmladdnormallinkfoot{G95}{http://www.g95.org/} (\mods{g95}),
which some feel is farther along in its development than \mods{gfortran}.
G95, however, is not supported as an Ubuntu package, and so I had
to manually install it.

I downloaded the binary version of G95 v0.91 
(the Linux x86\_64/EMT64 with 32 bit default integers) 
using the following
\cmd{curl} command:\footnote%
	{I use \mods{curl} because I usually access my
	Ubuntu server via a terminal session.}

\begin{codeblock}
\codeblockfont{%
\small
curl -o g95.tgz http://ftp.g95.org/v0.91/g95-x86\_64-32-linux.tgz}
\end{codeblock}

which saves the \fn{.tgz} file as the local file \fn{g95.tgz}.
After that, I followed the G95 project's standard
\latexhtml{installation instructions\footnote%
	{http://g95.sourceforge.net/docs.html\#starting}}%
	{\htmladdnormallink{installation instructions}%
		{http://g95.sourceforge.net/docs.html#starting}}
to finish the install.\footnote%
	{The G95 installation instructions say you can put
	\fn{g95-install} anywhere, and make a link to the
	executable \mods{g95} in
	\fn{$\sim$/bin}.  I put \fn{g95-install} in
	\fn{/usr/local}, and while in \fn{/usr/local/bin}, 
	I put a link to the G95 executable using the command:
	\begin{codeblock}
	\codeblockfont{%
	sudo ln -s ../g95-install\_64/bin/x86\_64-suse-linux-gnu-g95 g95.}
	\end{codeblock}}
The regular Linux x86 version of G95
(in \fn{g95-x86-linux.tgz} from the G95 website) did not work on my
machine.




%------------------------------------------------------------------------
\subsection{NetCDF Libraries}   \label{sec:ubuntu.netcdf}

%Here things were very confusing for my machine, as I needed to
%install two versions of the
%\htmladdnormallinkfoot{netCDF}%
%	{http://www.unidata.ucar.edu/software/netcdf/}
%libraries and include files, one 
%for a successful compilation of the extension modules
%(as described in Section~\ref{sec:create.so}),
%and the other 
%for a successful run of the pure-Fortran QTCM1 model
%(used to create the testing benchmarks, as described in
%Section~\ref{sec:test.qtcm}).
%
%The first set of netCDF files (for the extension modules) are
%installed from Ubuntu's package management system.
%These are automatically installed when the \mods{python-netcdf}
%package is installed via an Ubuntu package manager
%(see Section~\ref{sec:ubuntu.install.summary}).
%The include files for this netCDF installation are 
%located in \fn{/usr/include}, and the libraries for this
%netCDF installation are location in \fn{/usr/lib}.

For some reason, the netCDF libraries and include files
installed from the Ubuntu packages do not
correspond to the files needed
by the calling routines in \mods{qtcm}.  To solve this, I compiled
my own set of
\htmladdnormallinkfoot{netCDF 3.6.2 libraries}%
        {http://www.unidata.ucar.edu/software/netcdf/}
using the tarball
\latexhtml{downloaded from UCAR\footnote{http://www.unidata.ucar.edu/downloads/netcdf/netcdf-3\_6\_2/}}%
        {\htmladdnormallink{downloaded from UCAR}{http://www.unidata.ucar.edu/downloads/netcdf/netcdf-3_6_2/}}.

Once I uncompressed and untarred the package, and went into
the top-level directory of the package, I built the package by typing
the following at the Unix prompt:

\begin{codeblock}
\codeblockfont{%
export FC=g95 \\
export FFLAGS="-O -fPIC" \\
export FFLAGS="-fPIC" \\
export F90FLAGS="-fPIC" \\
export CFLAGS="-fPIC" \\
export CXXFLAGS="-fPIC" \\
./configure \\
make check \\
sudo make install}
\end{codeblock}

(The \cmd{export} commands set environment variables for the
Fortran compiler and Fortran and other compiler flags.  The
\vars{-fPIC} flag enables the compilers to create
position independent code, needed for shared libraries in
Ubuntu on a 64 bit Intel processor.)

The above installs the netCDF binaries, libraries, and include files into
sub-directories \fn{bin}, \fn{lib}, and \fn{include} in 
\fn{/usr/local}, the default.
The include files for this netCDF installation are thus
located in \fn{/usr/local/include}, and the libraries for this
netCDF installation are location in \fn{/usr/local/lib}.
(If you want to specify a different installation
location, use the \vars{--prefix} option in \cmd{configure}.)
While you don't have to have root privileges during the configuration
and check steps, you do during the installation step if you're installing
into \fn{/usr/local} (thus the \cmd{sudo} in the last step).\footnote%
	{Note that when you build netCDF, make sure the build directory
	is not in the directory tree of \vars{--prefix}
	or the default directory \fn{/usr/local}.}

%Because there are two different netCDF installations used in the
%\mods{qtcm} package, the makefiles for creating the benchmarks
%and extensions files will have different \vars{NCLIB} and \vars{NCINC}
%environment variables (see Section~\ref{sec:ubuntu.makefile}).




%------------------------------------------------------------------------
\subsection{Makefile Configuration}  \label{sec:ubuntu.makefile}

	\subsubsection{NetCDF}

In the \fn{src} directory in the \mods{qtcm} distribution, there is a
sub-directory \fn{Makefiles} that contains the makefiles for a
variety of platforms.  Edit the file \fn{makefile.ubuntu\_64\_g95}
so that the lines specifying the environment variables for the
netCDF libraries and include files:

\begin{codeblock}
\codeblockfont{%
NCINC=-I/usr/local/include \\
NCLIB=-L/usr/local/lib -lnetcdf}
\end{codeblock}

are changed to the path where your manually compiled
netCDF libraries and include files are.

Copy \fn{makefile.ubuntu\_64\_g95} from the \fn{Makefiles} sub-directory
in \fn{src} into \fn{src}.  
In other words, from the \mods{qtcm} distribution directory
(i.e., \fn{/buildpath}), at the Unix prompt execute:

\begin{codeblock}
\codeblockfont{%
cp src/Makefiles/makefile.ubuntu\_64\_g95 src/makefile}
\end{codeblock}


	\subsubsection{Linking Order}

Compilers in the GNU Compiler Collection (GCC) search libraries
and object files in the order they are listed in the command-line,
\latexhtml{from left-to-right\footnote%
	{http://gcc.gnu.org/onlinedocs/gcc-4.1.2/gcc/Link-Options.html\#index-l-670}}%
	{\htmladdnormallink{from left-to-right}{http://gcc.gnu.org/onlinedocs/gcc-4.1.2/gcc/Link-Options.html#index-l-670}}.
Thus, if routines in \fn{b.o} call routines in \fn{a.o}, 
you must list the files in the order \fn{a.o b.o}.

For some reason, that isn't the case for \mods{g95}.  Thus, you will
find \mods{g95} makefile rules structured like the following
(below is part of the rule to create an executable (\fn{qtcm}) for
benchmark runs):

% --- Two versions of this rule, one for display in PDF and the other
%     for display in HTML:
%
\begin{latexonly}
\begin{codeblock}
\codeblockfont{%
qtcm: main.o \\
\hspace*{8ex}\$(FC)~-O~\$(NCINC)~-o~\$@ main.o~\$(QTCMLIB)~\$(NCLIB)}
\end{codeblock}
\end{latexonly}

\begin{htmlonly}
\begin{rawhtml}
<p><code><font color="blue">qtcm: main.o<br>
&nbsp;&nbsp;&nbsp;&nbsp;&nbsp;&nbsp;&nbsp;$(FC) -O $(NCINC) -o 
$@ main.o $(QTCMLIB) $(NCLIB)</font></code></p>
\end{rawhtml}
\end{htmlonly}

even though \fn{main.o} depends on the QTCM library 
(specified in macro setting \vars{QTCMLIB}), which in turn
depends on the netCDF library (specified in macro setting \vars{NCLIB}).


	\subsubsection{Shared Object PIC}   \label{sec:sopic}

In order to compile the model in Ubuntu on a 64 bit Intel processor,
the model and the netCDF library it is linked to needs to be
compiled to be 
\latexhtml{position independent code (PIC).\footnote%
		{http://www.gentoo.org/proj/en/base/amd64/howtos/index.xml?part=1\&chap=3}}%
	{\htmladdnormallink{position independent code (PIC)}%
		{http://www.gentoo.org/proj/en/base/amd64/howtos/index.xml?part=1&chap=3}.}
This is accomplished with the 
\htmladdnormallinkfoot{\cmd{-fPIC} flag}%
	{http://www.fortran-2000.com/ArnaudRecipes/sharedlib.html}.

In the \mods{qtcm} makefiles, the \cmd{-fPIC} flag is introduced in the
macro \vars{FFLAGSM}, for instance:
\begin{codeblock}
\codeblockfont{%
FFLAGSM = -O -fPIC}
\end{codeblock}
For makefiles used in creating extension modules, \cmd{-fPIC} must
be passed into the \mods{f2py} call.  To do so, put the flags:
\begin{codeblock}
\codeblockfont{%
--f90flags="-fPIC" --f77flags="-fPIC"}
\end{codeblock}
after the \vars{--fcompiler} flag in the \mods{f2py} calling line.

The \cmd{-fPIC} flag must also be used when compiling the netCDF
libraries, as described in Section~\ref{sec:ubuntu.netcdf}.
Failure to create PIC libraries in 64 bit Ubuntu can result in errors 
like the following when creating the \mods{qtcm} extension modules:
\begin{codeblock}
\codeblockfont{%
ld: /usr/local/lib/libnetcdf.a(fort-attio.o): relocation R\_X86\_64\_32 against `a local symbol' can not be used when making a shared object; recompile with -fPIC /usr/local/lib/libnetcdf.a: could not read symbols: Bad value}
\end{codeblock}




%------------------------------------------------------------------------
\subsection{Summary of Steps}      \label{sec:ubuntu.install.summary}

The following summarizes all the steps I took to install
\mods{qtcm} in
Ubuntu 8.04.1 LTS (Hardy) running on a
Quad Core Intel Xeon (64 bit) machine.
Note that while I use the \mods{aptitude} package manager, you are
free to use any manager of your choice (e.g., \mods{apt-get},
\mods{synaptic}, etc.):

\begin{enumerate}
\item Install the G95 Fortran compiler from the binary distribution.
	See Section~\ref{sec:ubuntu.fort.install} for details.

\item Use an Ubuntu package manager
	to install the following packages, by typing:
	\begin{codeblock}
	\codeblockfont{%
sudo aptitude update \\
sudo aptitude install curl \\
sudo aptitude install python-epydoc \\
sudo aptitude install python-matplotlib \\
sudo aptitude install python-netcdf \\
sudo aptitude install python-scientific \\
sudo aptitude install python-scipy \\
sudo aptitude install texlive}
	\end{codeblock}

	Installing \mods{python-scipy} will also install NumPy and
	\mods{f2py}, so you don't have to install the
	\mods{python-numpy} package separately.

	Early-on as I debugged my \mods{qtcm} install on Ubuntu,
	I encountered errors that I thought came from an 
	\htmladdnormallinkfoot{old version of NumPy}%
		{http://cens.ioc.ee/pipermail/f2py-users/2008-June/001617.html},
	and thus I replaced Ubuntu's packaged NumPy with NumPy 1.1.0
	built 
	\latexhtml{directly from source.\footnote%
			{http://sourceforge.net/project/showfiles.php?group\_id=1369\&package\_id=175103}}%
		{\htmladdnormallink{directly from source}{http://sourceforge.net/project/showfiles.php?group_id=1369&package_id=175103}.}
	(Note, you shouldn't install your new NumPy in the default
	location, which may cause problems later-on with Ubuntu's
	package manager.)
	Later on, I concluded the errors I had encountered were not
	because of the NumPy version, but by then I didn't want to
	try to reinstall NumPy again.
	So strictly speaking, the version of Numpy I used is not
	the one bundled with \mods{python-scipy}, but that shouldn't
	be a problem.

\item Manually install netCDF 3.6.2 from source
	(see Section \ref{sec:ubuntu.netcdf}).

\item Manually install the \mods{basemap} package of
	\mods{matplotlib}.  
	The source for the \mods{basemap} toolkit is
	available 
	\latexhtml{from Sourceforge\footnote%
			{http://sourceforge.net/project/showfiles.php?group\_id=80706}}%
		{\htmladdnormallink{from Sourceforge}%
			{http://sourceforge.net/project/showfiles.php?group_id=80706}}
	I obtained version 0.9.9.1 using the
	following \cmd{curl} command:
	\begin{codeblock}
	\codeblockfont{%
\scriptsize
curl -o basemap.tar.gz $\backslash$ \\
http://voxel.dl.sourceforge.net/sourceforge/matplotlib/basemap-0.9.9.1.tar.gz}
	\end{codeblock}

	The \fn{README} file in the \fn{basemap-0.9.9.1} directory has
	detailed installation instructions.  Note that you have to
	install the GEOS library first (\fn{README} has detailed
	directions on how to do that too).  To be on the safe-side,
	I would set the \vars{FC} environment variable to the G95
	compiler
	(e.g., with \cmd{export FC=g95} in Bash).

\item From this point on, you can follow the
	general instructions given in Section~\ref{sec:install.sum},
	starting with step~\ref{list:download.qtcm.sum}.
	Please do not ignore, however, Section~\ref{sec:install.ubuntu}'s
	Ubuntu-specific details.

\end{enumerate}



% ===== end of file =====
