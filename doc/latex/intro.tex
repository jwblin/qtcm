%=====================================================================
% Introduction
%=====================================================================


% ----- BEGIN TEXT -----
%
%---------------------------------------------------------------------
\section{How to Read This Manual}

\emphpara{Most users:} 
Just read 
(1) the installation instructions in Chapter~\ref{ch:install},
(2) Chapter~\ref{ch:getting.started},
which tells you all you need to get started using \mods{qtcm}, and
(3) examples in Section~\ref{sec:cookbook} that give a feel
for how you can use the model.

\emphpara{Users having problems:}
Chapter~\ref{ch:trouble} provides troubleshooting tips for
a few problems.
The detailed description of how the package functions, 
in Chapter~\ref{ch:using}, will probably be more useful.

\emphpara{Developers:}
If you want to change the source code, please read
Chapter~\ref{ch:devnotes}.  Chapter~\ref{ch:future} describes
all the things I'd like to do to improve the package, but haven't
gotten to yet.




%---------------------------------------------------------------------
\section{About the Package}

The single-baroclinic mode
Neelin-Zeng Quasi-Equilibrium Tropical Circulation Model
\latexhtml{(QTCM1)\footnote{http://www.atmos.ucla.edu/$\sim$csi}}%
	{\htmladdnormallink{(QTCM1)}{http://www.atmos.ucla.edu/~csi}}
is a primitive equation-based intermediate-level atmospheric model
that focuses on simulating the tropical atmosphere.  Being more
complicated than a simple model, the model has full non-linearity
with a basic representation of baroclinic instability,
includes a radiative-convective feedback package, and includes a
simple land soil moisture routine (but does not include topography).
A brief, but more detailed, description of QTCM1 is given in
Section~\ref{sec:brief_qtcm}.

\htmladdnormallinkfoot{Python}{http://www.python.org}
is an interpreted, object-oriented, multi-platform,
open-source language that is used in a variety of software applications,
ranging from game development to bioinformatics.
In climate studies, Python has been used as the core language for
data analysis
(e.g., \htmladdnormallinkfoot{Climate Data Analysis Tools}{http://cdat.sf.net}),
visualization
(e.g., \htmladdnormallinkfoot{Matplotlib}{http://matplotlib.sf.net}),
and 
modeling
(e.g., \htmladdnormallinkfoot{PyCCSM}{http://code.google.com/p/pyccsm/}).

In comparison to traditional compiled languages like Fortran,
Python's lack of a separate compile step greatly simplifies the
debugging and testing phases of development, because code snippets
can be testing as code is written.
Python's extensive suite of higher-level tools (e.g., statistics,
visualization, string and file manipulation) accessible at runtime 
enables modeling and analysis to occur concurrently.  

The \mods{qtcm} package is an implementation of the Neelin-Zeng
QTCM1 in a Python object-oriented environment.  The conversion
package
\htmladdnormallinkfoot{\mods{f2py}}{http://cens.ioc.ee/projects/f2py2e/} is
used to wrap the QTCM1 Fortran model routines and manage model
execution using Python datatypes and utilities.  The result is a
modeling package where order and choice of subroutine execution can
be altered at runtime.  Model analysis and visualization can also
be integrated with model execution at runtime.




%---------------------------------------------------------------------
\section{Conventions In This Manual}

	\subsection{Audience}

In this manual I assume you have a rudimentary knowledge of Python.
Thus, I do not describe basic Python data types (e.g., dictionaries,
lists), object framework and syntax (e.g., classes, methods,
attributes, instantiation), module and package importing.  If you
need to brush up (or learn) Python, I'd recommend the following
resources:

\begin{itemize}
\item \htmladdnormallinkfoot{Python Tutorial:}{http://docs.python.org/tut/}
	This tutorial was written by Guido van Rossum, Python's original
	author.

\item \htmladdnormallinkfoot{Instant Hacking:}%
	{http://www.hetland.org/python/instant-hacking.php}
	Learn how to program with Python.

\item \htmladdnormallinkfoot{Dive Into Python:}%
	{http://diveintopython.org/index.html}
	Reasonably complete and cohesive. The entire book is available for 
	free online.

\item \htmladdnormallinkfoot{Handbook of the Physics Computing Course:}%
	{http://www.pentangle.net/python/handbook/}
	Written for a science audience.  Recommended.

\item \latexhtml{CDAT/Python Tips for Earth Scientists:\footnote%
	{http://www.johnny-lin.com/cdat\_tips/}}%
	{\htmladdnormallink{CDAT/Python Tips for Earth Scientists:}%
		{http://www.johnny-lin.com/cdat_tips/}}
	This web site is a FAQ of sorts for people using Python and
	the Climate Data Analysis Tools (CDAT) in the earth sciences,
	and thus focuses on using Python to do science rather than
	the computer science aspects of the language.

\end{itemize}

The purpose of this package is to make the QTCM1 model easier to
use.  In order to interpret the results, however, you still need
to have a robust sense of what climate models can and cannot tell
you.  A starting point for the QTCM1 model is the brief description
of the model in Section~\ref{sec:brief_qtcm}.  After that, I would
read the original papers describing the model formulation and results
\cite{Neelin/Zeng:2000,Zeng/etal:2000}, and 
\latexhtml{papers citing the model formulation work.\footnote%
{http://scholar.google.com/scholar?hl=en\&lr=\&cites=14217886709842286738}}%
{\htmladdnormallink{papers citing the model formulation work}%
{http://scholar.google.com/scholar?hl=en&lr=&cites=14217886709842286738}.}
Being an intermediate-level model using the quasi-equilibrium assumption,
QTCM1 (and thus \mods{qtcm}) has distinct strengths and limitations; 
please be aware of them.


	\subsection{Typographic Conventions}

\begin{center}
\begin{tabular}{\tabletabulardims}
\cmd{commands} & to be typed at the command-line
	are rendered in a 
	blue, serif, fixed-width typewriter font
	(e.g., \cmd{make \_qtcm\_full\_365}). \\ \hline
\dumarg{dummy arguments} &
	coupled with code or screen display is rendered in a 
	serif, proportional, italicized font
	(e.g., \screen{Error-Value too long in} \dumarg{variable}). \\ \hline
\fn{file and directory names} & are rendered in a 
	sans-serif, italicized font
	(e.g., \fn{setbypy.F90}). \\ \hline
\screen{screen display} & is rendered in a 
	green, serif, fixed-width typewriter font. \\ \hline
\mods{module, method, and subroutine names} & are rendered in a 
	blue, serif, fixed-width typewriter font. \\ \hline
\vars{variable and attribute names} & are rendered in a 
	blue, serif, fixed-width typewriter font. \\ \hline
\class{class names} & are rendered in a 
	blue, serif, fixed-width typewriter font.
\end{tabular}
\end{center}

Blocks of code (usually commands, screen display, and module,
variable, and class names) are displayed in a blue, serif, fixed-width
typewriter font.


	\subsection{Terminology}

\begin{description}
\item[attribute and method references:]
	If there is any possibility of confusion, I will give the
	class that an attribute or method comes from when that
	attribute or method is referenced.  If no class is mentioned
	by name or context,
	assume that the attribute/method comes from the
	\class{Qtcm} class.

\item[``compiled QTCM1 model'':]
	This usually is used to denote when I'm talking about
	compiled Fortran QTCM1 routines and variables therein,
	as an extension module to the Python \mods{qtcm} package..
	Thus, ``compiled QTCM1 model \vars{u1}'' is the value
	of variable \vars{u1} in the Fortran model, not the
	value at the Python-level.  Sometimes I refer to the
	compiled QTCM1 model as just ``QTCM1'' or as
	``compiled QTCM1 Fortran model''.

\item[``pure-Fortran QTCM1'':]
	This refers to the Neelin-Zeng QTCM1 model in it's
	original Fortran form, not as an extension module to
	the Python \mods{qtcm} package.

\item[``Python-level'':]
	This usually denotes the value of a variable as an
	attribute of a \class{Qtcm} instance.  This variable
	is stored at the Python interpreter level.

\item[\class{Qtcm}:]
	This refers to the Python \class{Qtcm} class
	(note the capitalized first letter).

\item[\mods{qtcm}:]
	This refers to the Python \mods{qtcm} package.

\item[QTCM1 vs.\ QTCM:]
	Although the QTCM1 is currently the only version of a
	quasi-equilibrium tropical circulation model (QTCM), in
	principle one can construct a QTCM with any number of
	baroclinic modes.  In anticipation of this, the names of
	the Python package and class do not end in a numeral.  In
	this manual and the \mods{qtcm} docstrings, QTCM and QTCM1
	are used interchangably.
	Usually either of these phrases mean the quasi-equilibrium
	tropical circulation model in a generic sense, regardless
	of its form of implementation.
\end{description}




%---------------------------------------------------------------------
\section{Current Version Information and Acknowledgments}  \label{sec:ver}

% This file is automatically generated by
    % code_to_latex.py.

This manual describes version 0.1.2 (dated September 12, 2008), of package \mods{qtcm}.
\input{pkg_author.tex}is the primary author of the package.

The \mods{qtcm} package is built upon the pure-Fortran QTCM1 model,
version 2.3 (August 2002), with a few minor changes.
Those changes are described in detail in
Section~\ref{sec:f90changes}.

The homepage for the \mods{qtcm} package is
\htmladdnormallink{http://www.johnny-lin.com/py\_pkgs/qtcm}%
	{http://www.johnny-lin.com/py_pkgs/qtcm}.
All Python code in this package, 
and the Fortran files \fn{setbypy.F90} and \fn{wrapcall.F90},
are \copyright\ 2003--2008 by 
\htmladdnormallinkfoot{Johnny Lin}%
		{http://www.johnny-lin.com} 
and constitutes a
library that is covered under the GNU Lesser General Public License
(LGPL):

\begin{quotation}
	This library is free software; you can redistribute it
	and/or modify it under the terms of the 
	\htmladdnormallinkfoot{GNU Lesser General Public License}%
		{http://www.gnu.org/copyleft/lesser.html} 
	as published by
	the Free Software Foundation; either version 2.1 of the
	License, or (at your option) any later version.

	This library is distributed in the hope that it will be
	useful, but WITHOUT ANY WARRANTY; without even the implied
	warranty of MERCHANTABILITY or FITNESS FOR A PARTICULAR
	PURPOSE. See the GNU Lesser General Public License for more
	details.

	You should have received a copy of the GNU Lesser General
	Public License along with this library; if not, write to
	the Free Software Foundation, Inc., 59 Temple Place, Suite
	330, Boston, MA 02111-1307 USA.

	You can contact Johnny Lin at his email address 
	or at North Park University, Physics Department,
	3225 W. Foster Ave., Chicago, IL 60625, USA.  
\end{quotation}

All other Fortran code in this package, as well as the makefiles,
are covered by licenses (if any) chosen by their respective owners.

This manual, in all forms (e.g., HTML, PDF, \LaTeX),
is part of the documentation of the \mods{qtcm} package 
and is \copyright\ 2007--2008 by Johnny Lin.
Permission is granted to copy, distribute and/or modify 
this document under the terms of the 
GNU Free Documentation License, Version 1.2 
or any later version published by the Free Software Foundation; 
with no Invariant Sections, no Front-Cover Texts, 
and no Back-Cover Texts. 
A copy of the license can be found 
\htmladdnormallinkfoot{here}{http://www.gnu.org/licenses/fdl.html}.

Transparent copies of this document are located online in
\latexhtml{%
\htmladdnormallinkfoot{PDF}%
	{http://www.johnny-lin.com/py\_pkgs/qtcm/doc/manual.pdf}}%
{\htmladdnormallink{PDF}%
	{http://www.johnny-lin.com/py_pkgs/qtcm/doc/manual.pdf}}
and
\latexhtml{%
\htmladdnormallinkfoot{HTML}%
	{http://www.johnny-lin.com/py\_pkgs/qtcm/doc/}}%
{\htmladdnormallink{HTML}%
	{http://www.johnny-lin.com/py_pkgs/qtcm/doc/}}
formats.
The \LaTeX\ source files are distributed with the \mods{qtcm}
distribution.
While the HTML version is nearly identical to the PDF
and \LaTeX\ versions, not every feature in the manual was successfully
converted.  This is particularly true with figures, which are
unnumbered in the HTML version and may be formatted differently
than the authoritative PDF version.
Thus, please consider the \LaTeX\ version as the authoritative
version.

\vspace{\baselineskip}

\emphpara{Acknowledgements:}
Thanks to David Neelin and Ning Zeng and the Climate Systems
Interactions Group at UCLA for their encouragement and help.
On the Python side,
thanks to Alexis Zubrow, Christian Dieterich, Rodrigo Caballero,
Michael Tobis, and Ray Pierrehumbert for steering me straight.
Early versions of some of this work was carried out 
at the University of Chicago Climate Systems Center, 
funded by the National Science Foundation (NSF) 
Information Technology Research Program under grant ATM-0121028. 
Any opinions, findings and conclusions or recommendations 
expressed in this material are those of the author and 
do not necessarily reflect the views of the NSF.

Intel\textregistered\ and
   Xeon\textregistered\ are registered trademarks of Intel Corporation.
Matlab\textregistered\ is a registered trademark of The MathWorks.
UNIX\textregistered\ is a registered trademark of The Open Group.




%---------------------------------------------------------------------
\section{Summary of Release History}

\begin{itemize}
\item 2008 Sep 12:  Version 0.1.2 released.  Summary of changes:
	\begin{itemize}
	\item Create \class{Qtcm} method \mods{get\_qtcm1\_item}.
		This method is effectively an alias of method 
		\mods{get\_qtcm\_item}.
	\item Create \class{Qtcm} method \mods{set\_qtcm1\_item}.
		This method is effectively an alias of method 
		\mods{set\_qtcm\_item}.
	\item Update User's Guide to phase out references to
		the \mods{get\_qtcm\_item}
		and \mods{set\_qtcm\_item} methods.  
		Adding the ``1'' to the method names makes the purpose
		of the methods clearer.
	\item Add unit tests to cover methods \mods{get\_qtcm1\_item} and
		\mods{set\_qtcm1\_item}.
	\end{itemize}

\item 2008 Jul 30:  Updates to the User's Guide (just the online versions,
        not the copies released with the tarball).

\item 2008 Jul 15:  First publicly available distribution 
	released (v0.1.1).
\end{itemize}




%---------------------------------------------------------------------
\section{A Brief Description of The QTCM1}   \label{sec:brief_qtcm}

This description is copied from Ch.\ 3 of Lin \cite{Lin:2000}, 
with minor revisions.
Model formulation is fully described in
Neelin \& Zeng \cite{Neelin/Zeng:2000} and model
results are described in Zeng et~al.\ \cite{Zeng/etal:2000}.
Neelin \& Zeng \cite{Neelin/Zeng:2000} is based upon v2.0 of QTCM1
and Zeng et~al.\ \cite{Zeng/etal:2000} is based on QTCM1 v2.1.
The 
\latexhtml{%
\htmladdnormallinkfoot{QTCM1 manual}%
	{http://www.atmos.ucla.edu/$\sim$csi/qtcm\_man/v2.3/qtcm\_manv2.3.pdf}}%
{\htmladdnormallink{QTCM1 manual}%
	{http://www.atmos.ucla.edu/~csi/qtcm_man/v2.3/qtcm_manv2.3.pdf}}
\cite{Neelin/etal:2002}
describes the details of model implementation.

QTCM1 differs from most full-scale GCMs primarily in how the vertical
temperature, humidity, and velocity structure of the atmosphere is
represented.  First, instead of representing the vertical structure
by finite-differenced levels, the model uses a Galerkin expansion
in the vertical.  The vertical basis functions are chosen according
to analytical solutions under convective quasi-equilibrium conditions,
so only a few need be retained.
Temperature and humidity are each described by separate
vertical basis functions ($a_1$ and $b_1$, respectively).
Low-level variations in the humidity basis
are larger than in the temperature basis.
For velocity, QTCM1 uses a single baroclinic basis function ($V_1$)
defined consistently with the temperature basis function,
as well as a barotropic velocity mode ($V_0$).
The vertical profiles of $a_1$, $b_1$, and $V_1$
are given in Figure~\ref{fig:qtcm.basis}.
Currently, QTCM1 does not include a separate
vertical degree of freedom describing the PBL.
The horizontal grid spacing of the model is 
$5.625^{\circ}$ longitude by $3.75^{\circ}$ latitude.


% <QTCM1 beta version vertical structure modes>
%
% (1) LaTeX version:
%
\begin{latexonly}
\begin{figure}
   \noindent
   \begin{minipage}[b]{.49\linewidth}
      \settowidth{\enumlabel}{(a) }%
      \setlength{\remainder}{\linewidth}% 
      \addtolength{\remainder}{-\enumlabel}
      {(a)}~\makebox[\remainder]{$a_1$ and $b_1$}
      \centering\includegraphics[width=\linewidth,viewport=58 72 389 344,clip]%
                    {figs/a1b1.pdf}
   \end{minipage}\hfill
   \begin{minipage}[b]{.49\linewidth}
      \settowidth{\enumlabel}{(b) }%
      \setlength{\remainder}{\linewidth}% 
      \addtolength{\remainder}{-\enumlabel}
      {(b)}~\makebox[\remainder]{$V_1$}

      \centering\includegraphics[width=\linewidth,viewport=58 72 389 346,clip]%
                    {figs/V1.pdf}
   \end{minipage}

   \caption{Vertical profiles of basis functions for
		(a) temperature $a_1$ (solid) and humidity $b_1$ (dashed) and
		(b) baroclinic component of
		horizontal velocity $V_1$.}
   \label{fig:qtcm.basis}
\end{figure}
\end{latexonly}

% (2) HTML replacement version:
%
\begin{htmlonly}
\label{fig:qtcm.basis}
\begin{center}
\htmladdimg{../latex/figs/a1b1.png}
\htmladdimg{../latex/figs/V1.png}

\htmlfigcaption{Figure \ref{fig:qtcm.basis}:  
	Vertical profiles of basis functions for
   	(a) temperature $a_1$ (solid) and humidity $b_1$ (dashed) and
   	(b) baroclinic component of
   	horizontal velocity $V_1$.}
\end{center}
\end{htmlonly}


These modes are chosen to accurately capture deep convective regions.
Outside deep convective regions the mode
is simply a highly truncated
Galerkin representation.  The system is much more tightly constrained than
a full-scale GCM, yet hopefully retains the essential dynamics and nonlinear
feedbacks.  The result is that QTCM1 is easier to diagnose than a GCM,
and is computationally fast (about 8 minutes per year on a Sun Ultra 2
workstation).  Zeng et al.\ \cite{Zeng/etal:2000} show results indicating
this intermediate-level model does a reasonable job simulating
tropical climatology and ENSO variability.  


Below is a summary of the main model equations \cite{Neelin/Zeng:2000}:
\begin{equation}
   \partial_t \mathbf{v}_1 
      + \D_{V1} (\mathbf{v}_0 , \mathbf{v}_1)
      + f \mathbf{k} \times \mathbf{v}_1
      =
   - \kappa \nabla T_1 
      - \epsilon_1 \mathbf{v}_1 
      - \epsilon_{01} \mathbf{v}_0
   \label{eqn:barocl_wind}
\end{equation}
\begin{equation}
   \partial_t \zeta_0 
      + \mathrm{curl}_z (\D_{V0} (\mathbf{v}_0 , \mathbf{v}_1))
      + \beta v_0
      =
   - \mathrm{curl}_z (\epsilon_0 \mathbf{v}_0)
      - \mathrm{curl}_z (\epsilon_{10} \mathbf{v}_1)
   \label{eqn:barotr_wind}
\end{equation}
\begin{equation}
   \aonehat (\partial_t + \D_{T1}) T_1 
      + M_{S1} \nabla \cdot {\bf v}_1 
      =
   \langle Q_c \rangle
      + (g/p_T) (-R^\uparrow_t -R^\downarrow_s + R^\uparrow_s + S_t - S_s + H)
   \label{eqn:temperature}
\end{equation}
\begin{equation}
   \bonehat (\partial_t + \D_{q1}) q_1 
      - M_{q1} \nabla \cdot {\bf v}_1 
      =
   \langle Q_q \rangle
      + (g/p_T) E
   \label{eqn:moisture}
\end{equation}
where (\ref{eqn:barocl_wind}) describes the baroclinic wind component,
      (\ref{eqn:barotr_wind}) describes the barotropic wind component,
      (\ref{eqn:temperature}) is the temperature equation, and
      (\ref{eqn:moisture}) is the moisture equation.

In the simplest formulation, the vertically integrated
convective heating and moisture sink
are assumed to be equal and opposite, so:
\begin{equation}
  -\langle Q_q \rangle = \langle Q_c \rangle 
                              = \epsilon^\ast_c (q_1 - T_1)
\end{equation}

For its convective parameterization for $Q_c$, this model uses the
Betts-Miller \cite{Betts/Miller:1986} moist convective
adjustment scheme, a scheme that is also used in some GCMs.
In the convective parameterization, the coefficient
$\epsilon^\ast_c$ is defined as:
\begin{equation}
   \epsilon^\ast_c 
      \equiv 
   \aonehat \bonehat (\aonehat + \bonehat)^{-1} \tau_c^{-1} 
      \mathcal{H}( \mathit{C}_{\mathrm{1}} )
\end{equation}
where $\mathcal{H}( \mathit{C}_{\mathrm{1}} )$ is zero for
$C_{1} \leq 0$, and one for $C_{1} > 0$, and $C_{1}$
is a measure of the convective available potential energy (CAPE),
projected onto the model's temperature and moisture basis functions.

Sensible heat ($H$) and evaporation ($E$) are given as
bulk-aerodynamic formulations:
\begin{equation}
   H
      =
   \rho_a C_D \mathrm{V}_s (T_s - (T_{rs} + a_{1s} T_1))
\end{equation}
\begin{equation}
   E
      =
   \rho_a C_D \mathrm{V}_s (q_\mathit{sat} (T_s) 
      - (q_{rs} + b_{1s} q_1))
\end{equation}

Longwave radiation components are denoted by $R$, and net shortwave
radiation is denoted by $S$.
The terms $\D_{V1}$ and $\D_{V0}$ are the advection-diffusion operators
for the momentum equations (projected onto $V_0$ and $V_1 (p)$,
respectively).
The terms $\D_{T1}$ and $\D_{q1}$ are the
advection-diffusion operators for the temperature and moisture
equations, respectively, using a vertical average projection.
The $\langle X \rangle$ and $\widehat{X}$ operators are
equivalent and denote vertically integration over the troposphere.
Please see Neelin \& Zeng \cite{Neelin/Zeng:2000} and 
Zeng et al.\ \cite{Zeng/etal:2000}
for a more complete description of equations and coefficients.







% ====== end file ======
